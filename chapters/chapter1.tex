\chapter{Einleitung \label{cha:chapter1}}

In der heutigen Zeit ist Künstliche Intelligenz (\gls{KI}) nicht
mehr aus unserem täglichen Leben wegzudenken. 
Sie begegnet uns im Auto, beim Musicstreaming oder der Navigation ganz unbewusst. 
Spätestens seit dem Release von ChatGPT am 30. November 2022, 
einer KI basierend auf umfangreichen Sprachmodellen, 
welche eine interaktive Kommunikation ermöglicht, 
\footcite{ChatGPT} ist KI zu einer 
der wichtigsten Innovationen dieses Jahrhunderts aufgestiegen. 
Mit rund 1.8 Milliarde Nutzern im Monat April 2024 ist ChatGPT  
\footcite{NumberChatGPTUsers2023} 
und damit einhergehend das Thema KI auch aktiv in den Vordergrund des Bewusstseins 
der Allgemeinheit gerückt. 
Es lassen sich jedoch mittlerweile mithilfe von generativer KI 
nicht nur Text sondern auch neue 
Dateninstanzen verschiedener Art erzeugen \footcite{WasIstKuenstliche}. 
Bei der Generierung von neuartigen Werken, 
Erfindungen und Dateninstanzen durch KI entstehen so 
urheberrechtliche und 
patentrechtliche Fragestellungen die in dieser Arbeit von der 
patentrechtlichen Seite beleuchtet werden.


\section{Motivation\label{sec:moti}}

In den letzten Jahren hat die rasante Entwicklung 
der KI-Technologien zu einer neuen Ära der Innovation geführt. 
Künstliche Intelligenz ist in der Lage, komplexe Aufgaben zu bewältigen, 
die traditionell menschliche Kreativität und Intelligenz erfordern. 
\\

Das deutsche \gls{PatG} ist darauf ausgelegt, 
Erfindungen zu schützen, die von Menschen gemacht wurden
\footcite{DPMAPatentschutz}. 
Erfindungen durch KI stellen eine neue Herausforderung für 
das Patentrecht und die Prüfung von Patentanträgen dar, 
da es schwer ist zu definieren wer der „Erfinder“ bei KI-generierten
Innovationen ist und ob 
KI-generierte Werke generell die Kriterien der Patentierbarkeit erfüllen können. 
Unternehmen und Erfinder sind auf den Schutz von Innovationen durch 
Patente angewiesen, um Investitionen und Wettbewerbsvorteile zu sichern. 
Wenn KI-generierte Erfindungen nicht patentierbar sind, 
könnte dies Innovationen hemmen 
und Forschung sowie Entwicklung entschleunigen. 

\section{Zielsetzung\label{sec:objective}}

Das Ziel dieser Arbeit ist es, 
die Bedingungen und Herausforderungen zu untersuchen,
unter denen von KI geschaffene Computerprogramme im deutschen Patentrecht 
patentierbar sind. Hierbei liegt der Fokus auf den Aspekten der Neuheit, 
der Erfinderfrage, der erfinderischen Tätigkeit 
und den grundlegenden Bedingungen für die Patentierung von Computerprogrammen. 
\\

Diese Arbeit beleuchtet, 
was im Kontext von KI-generierten Computerprogrammen 
als neu betrachtet wird 
und wie Neuheit solcher Programme 
im Rahmen des deutschen Patentrechts beurteilt wird. 
Ein weiterer Punkt besteht darin, zu klären 
wer als Erfinder gilt,
wenn das Programm von einer KI erstellt wurde. 
Hier wird analysiert, 
ob und inwiefern eine KI selbst als Erfinder in Erscheinung treten kann 
oder ob der Mensch, der diese bedient, 
diese Rolle übernehmen kann oder sogar muss. 
Es wird zudem geprüft, 
wie die erfinderische Tätigkeit im Zusammenhang 
mit KI-generierten Programmen bewertet wird. 
Dies beinhaltet die Frage, 
ob und wie der kreative Beitrag einer KI in diesem Kontext zu beurteilen ist. 
Außerdem untersucht diese Arbeit die allgemeinen Voraussetzungen, 
unter denen Computerprogramme im deutschen Patentrecht patentierbar sind, 
und wie diese auf Programme, 
die von KI-Systemen erstellt wurden, angewendet werden können. 
\\

Zur Erreichung der Ziele dieser Arbeit werden Präzedenzfälle analysiert, 
juristische Fachliteratur herangezogen, sowie Gesetzestexte untersucht. 
Dabei werden Gerichtsurteile und Entscheidungen analysiert, 
die Aufschluss über die bisherige Vorgehensweise 
im Bereich der Patentierung von Computerprogrammen geben. 
Diese Arbeit stützt sich zudem auf bestehende juristische Fachliteratur, 
um die aktuellen Diskussionen 
und theoretischen Grundlagen zu diesem Thema darzustellen.
Gesetzestexte wie das deutsche PatG werden untersucht, 
um die formellen Voraussetzungen und rechtlichen Rahmenbedingungen darzustellen. 
Um die Historie der Gesetzgebung in Deutschland vollumfänglich aufzuarbeiten,
werden relevante Rechtssprechungen, welche für die heutige Gesetzgebung
maßgebend wahren hervorgehoben, sowie der Einfluss durch das Europäische
Patentamt.
\\
Ein weiterer Bestandteil der Arbeit ist die Entwicklung eines hypothetischen Patents 
für ein KI-generiertes Computerprogramm. 
Dieser Abschnitt der Arbeit umfasst eine detaillierte Beschreibung der Funktionsweise 
und der technischen Merkmale des von der KI generierten Computerprogrammes, 
die Formulierung von Patentansprüchen, 
sowie eine Schritt-für-Schritt-Darstellung des Prozesses, 
wie dieses Patent im aktuellen rechtlichen Rahmen angemeldet werden könnte, 
einschließlich der potenziellen Herausforderungen und Hürden. 
\\
Diese Arbeit zielt darauf ab, 
ein umfassendes Verständnis der rechtlichen 
und praktischen Aspekte der 
Patentierbarkeit von KI-generierten Computerprogrammen zu vermitteln 
und mögliche Lösungsansätze für die entstehenden
Herausforderungen aufzuzeigen. 


\section{Umfang\label{sec:scope}}

Diese Arbeit umfasst die Darstellung der rechtlichen
Rahmenbedingungen für die Patentierbarkeit von KI-generierten
Computerprogrammen im deutschen Patentrecht, sowie deren
praktische Anwendung.
Dabei sind vorallem die Paragraphen 1,3,4 und 37 des Patentgesetzes relevant
im Bezug auf KI-generierte Computerprogramme.
\\
Gemäß § 37 PatG muss bei der Anmeldung eines Patents 
ein Erfinder benannt werden. 
Ein zentraler Aspekt dieser Arbeit wird die Frage sein, 
wer rechtlich als Erfinder einer KI-generierten Software gilt. 
Dabei werden der Fall "DABUS"\footcite{zivilsenatZB222024} und die Rechtssprechungen
vom europäischen Patentamt dazu analysiert. Diese Rechtssprechungen
haben den Weg für das Vorgehen bei der Anmeldung von KI-Patenten, geebnet. 
Von dort ausgehend 
wird die rechtliche Lage in Deutschland untersucht und
Unterschiede, sowie Gemeinsamkeiten dargestellt. Für einen größeren Überblick
wird die Erfinderbenennung auch internationale
betrachtet.
\\
Die allgemeinen Voraussetzungen 
für die Patentierbarkeit von Computerprogrammen
schließen die Patentierbarkeit von Computerprogrammen "als solche" 
gemäß § 1 Abs. 3 PatG aus. 
Computerprogramme müssen, um patentiert werden zu können,
eine technische Wirkung aufweisen um nicht "als solche" zu gelten.
Dabei wird auf die Definition einer technischen Erfindung und Urteile
in Bezug auf Patente von Computerprogrammen eingegangen. Es
werden Urteile des EPA und BGH analysiert, welche
die Technizität von Computerprogrammen bewerten.
Durch die Analyse der Entstehung der rechtlichen Rahmenbedingungen
in Deutschland entsteht ein Bewusstsein für die 
Beurteilung der Technizität von Computerprogrammen.
Außerdem wird die Schutzwirkung von Patenten  
zu der anderer geistigen Eigentumsrechte wie dem Urheberrecht, 
gemäß dem Urheberrechtsgesetz (UrhG)abgegrenzt.
Das Urheberrecht bietet primär Schutz für schöpferische Werke,
während Patenten technische Erfindungen schützen.
\\
Gemäß § 3 PatG müssen Erfindungen neu sein, 
um patentierbar zu sein. 
Hier werden relevante Rechtssprechungen 
analysiert um die Neuheit von KI generierten
Erfindungen zu bestimmen. Insbesondere das
Urteil des BGH zum "Antiblockiersystem" ist 
hierbei entscheidend zur Einschätung, ob KI
etwas neues erfinden kann \footcite{Bundesgerichtshof13051980}.
\\
Gemäß § 4 PatG müssen Erfindungen auf einer erfinderischen Tätigkeit beruhen. 
Die Bewertung der erfinderischen Tätigkeit 
bei KI-generierten Programmen ist ebenfalls ein
zentraler Punkt dieser Arbeit. Hier wird zwischen
Patenten entschieden, welche den Einsatz von KI
als Schutzanspruch haben und jenen, in denen eine
KI selber patentierbares erzeugt hat.
\begin{figure}[htb]
  \centering
  \includegraphics[width=\textwidth]{img/Patentrecht Übersicht.pdf}\\
  \caption{Patentrecht Gesamtbild}\label{fig:patentrecht}
\end{figure}
Abbildung \ref{fig:patentrecht} zeigt die Zusammenhänge zwischen PatG, 
künstlicher Intelligenz und Computerprogrammen.
Das europäische Patentübereinkommen und der United States Patent Act 
werden in dieser Arbeit nicht weitergehen behandelt,
da der Fokus auf dem deutschen Patentrecht liegt.
Eine Abgrenzung erfolgt indirekt durch die unterschiedlichen
Rechtssprechungen der zugehörigen Patentämter, welche einen 
Einfluss auf die Rechtssprechungen in Deutschland haben.
\\


\section{Gliederung\label{sec:outline}}


\textbf{Kapitel \ref{cha:chapter2}} 
Dieses Kapitel bietet einen umfassenden Überblick über die Grundlagen
von KI, vorallem im Bezug auf generative KI, welche in der
Lage ist schutzfähige Werke zu erstellen.
Es erläutert den Aufbau, 
und die Funktionsweise generativer KI 
und die verschiedenen Arten von KI, sowie Typen von KI.
Grundlegende Terminologien wie Deep Learning, Neuronale Netze und
KI-Modelle werden erläutert um zu verstehen, inwieweit KI
in der Lage ist, schutzfähige Werke selbstständig zu erstellen.
\\
\textbf{Kapitel \ref{cha:chapter3}} 
In diesem Kapitel wird die Patentierbarkeit von Erfindungen untersucht,
die durch künstliche Intelligenz erstellt werden, 
sowie die Patentierbarkeit von Computerprogrammen im deutschen Patentrecht. 
Außerdem wird untersucht, wie Gerichte bisherige Fälle behandelt haben 
und welche Schlussfolgerungen daraus für die aktuelle Praxis gezogen werden können. 
Es befasst sich mit Anforderungen an die Neuheit 
und erfinderische Tätigkeit solcher Innovationen gemäß dem deutschen PatG. 
Besonderer Fokus liegt auf der Fragestellung, 
ob und wie KI als Erfinder rechtlich anerkannt werden kann 
und welche rechtlichen Herausforderungen dies mit sich bringt.
\\
\\
\textbf{Kapitel \ref{cha:chapter5}} 
Im Mittelpunkt dieses Kapitels steht die Entwicklung eines hypothetischen Patentantrags 
für ein KI-generiertes Computerprogrammm. 
Es bietet eine detaillierte Beschreibung technischer Merkmale von Computerprogrammen, 
formuliert Patentansprüche und 
skizziert den Prozess der Patentanmeldung im Rahmen des deutschen Rechtssystems. 
Potenzielle Herausforderungen und Lösungsansätze bei der Patentierung werden dargestellt, 
um Einblicke in den Anmeldeprozess zu geben.
\\
\\
\textbf{Kapitel \ref{cha:chapter6}} 
Dieses Kapitel zieht eine Bilanz der vorangegangenen Untersuchungen und Analysen. 
Es beleuchtet die ermittelten Ergebnisse in Bezug auf die rechtliche Bewertung 
von KI-generierten Computerprogrammen im deutschen Patentrecht. 
Dabei werden die wichtigsten Erkenntnisse herausgearbeitet 
und offene Fragen sowie potenzielle 
Weiterentwicklungen im Patentrecht dargestellt.
\\
\\
\textbf{Kapitel \ref{cha:chapter7}} 
Abschließend fasst das Kapitel Fazit und Ausblick die zentralen Ergebnisse zusammen 
und gibt einen Ausblick auf zukünftige Entwicklungen im Bereich der Patentierbarkeit 
von KI-generierten Computerprogrammen. 
Es hebt die Bedeutung der rechtlichen Klarstellungen hervor, 
die notwendig sind, um Innovationen im Bereich der künstlichen Intelligenz zu schützen, 
und stellt mögliche Ansätze für eine Weiterentwicklung des deutschen Patentrechts 
dar.