\chapter{Einleitung \label{cha:chapter1}}

In der heutigen Zeit ist Künstliche Intelligenz(KI) kaum noch aus unserem täglichen Leben wegzudenken. 
Sie begegnet uns im Auto, beim Musicstreaming oder der Navigation ganz unbewusst. 
Spätestens seid dem Release von ChatGPT am 30. November 2022, 
einer KI basierend auf umfangreichen Sprachmodellen, welche eine interaktive Kommunikation ermöglichen 
\cite{ChatGPT} ist künstliche Intelligenz zu einer der wichtigsten Innovationen dieses Jahrhunderts aufgestiegen. 
Mit rund 1.8 Milliarde Millionen Nutzern im Monat April 2024 \cite{NumberChatGPTUsers2023} 
von ChatGPT ist KI nun auch aktiv in den Vordergrund des Bewusstseins der Allgemeinheit gerückt. 
Darüberhinaus lassen sich mittlerweile mithilfe von generativer KI nicht nur Text sondern auch neue 
Dateninstanzen verschiedener Art erzeugen \cite{WasIstKuenstliche}. Bei der Generierung von neuartigen Werken, 
Erfindungen und Dateninstanzen durch künstliche Intelligenz entstehen so urheberrechtliche und 
patentrechtliche Fragestellungen die in dieser Arbeit von der patentrechtlichen Seite beleuchtet werden.

\section{Motivation\label{sec:moti}}

In den letzten Jahren hat die rasante Entwicklung der KI-Technologien zu einer neuen Ära der Innovation geführt. 
Künstliche Intelligenz ist in der Lage, komplexe Aufgaben zu bewältigen, 
die traditionell menschliche Kreativität und Intelligenz erfordern. 
\\

Das deutsche Patentgesetz ist darauf ausgelegt, 
Erfindungen zu schützen, die von Menschen gemacht wurden. 
\cite{DPMAPatentschutz}. Erfindungen durch KI stellen eine neue Herausforderungen dar, 
da es schwer ist, zu definieren wer der „Erfinder“ ist und ob 
KI-generierte Werke die Kriterien der Patentierbarkeit erfüllen. 
Unternehmen und Erfinder sind auf den Schutz von Innovationen durch 
Patente angewiesen, um Investitionen und Wettbewerbsvorteile zu sichern. 
Wenn KI-generierte Erfindungen nicht patentierbar sind, 
könnte dies Innovationen hemmen und Forschung sowie Entwicklung entschleunigen. 


\section{Zielsetzung\label{sec:objective}}

Das Ziel dieser Arbeit ist es, 
die Bedingungen und Herausforderungen zu untersuchen,
unter denen von KI geschaffene Computerprogramme im deutschen Patentrecht 
patentierbar sind. Hierbei liegt der Fokus auf den Aspekten der Neuheit, 
der Erfinderfrage, der erfinderischen Tätigkeit 
und den grundlegenden Bedingungen für die Patentierung von Computerprogrammen. 
\\

Diese Arbeit beleuchtet, 
was im Kontext von KI-generierten Computerprogrammen 
als neu betrachtet wird 
und wie Neuheit solcher Programme im Rahmen des deutschen Patentrechts beurteilt wird. 
Ein weiterer Punkt besteht darin, zu klären, 
wer als Erfinder gilt,
wenn das Programm von einer KI erstellt wurde. 
Hier wird analysiert, ob und inwiefern eine KI selbst als Erfinder in Erscheinung treten kann 
oder ob der Mensch, der die KI programmiert oder diese bedient, diese Rolle übernimmt. 
Es wird zudem geprüft, 
wie die erfinderische Tätigkeit im Zusammenhang mit KI-generierten Programmen bewertet wird. 
Dies beinhaltet die Frage, ob und wie der kreative Beitrag einer KI in diesem Kontext zu beurteilen ist. 
Außerdem untersucht diese Arbeit die allgemeinen Voraussetzungen, 
unter denen Computerprogramme im deutschen Patentrecht patentierbar sind, 
und wie diese auf Programme, die von KI-Systemen erstellt wurden, angewendet werden können. 
\\

Zur Erreichung der Ziele dieser Arbeit werden Präzedenzfällen analysiert, 
juristische Fachliteratur herangezogen, sowie Gesetzestexte untersucht. 
Dabei werden relevante Gerichtsurteile und Entscheidungen analysiert, 
die Aufschluss über die bisherigen Handhabungen 
und Interpretationen von Gesetzen im Bereich der Patentierung von Computerprogrammen geben. 
Zudem stützt sich die Arbeit auf bestehende juristische Fachliteratur, 
um die aktuellen Diskussionen und theoretischen Grundlagen zu diesem Thema darzustellen. 
Relevante Gesetzestexte werden untersucht, insbesondere das deutsche Patentgesetz, 
um die formellen Voraussetzungen und rechtlichen Rahmenbedingungen darzustellen. 
\\

Ein weiterer Bestandteil der Arbeit ist die Entwicklung eines hypothetischen Patents 
für eine KI-generierte Software. 
Dieser Abschnitt der Arbeit umfasst eine detaillierte Beschreibung der Funktionsweise 
und der technischen Merkmale der von der KI generierten Software, 
die Formulierung von Patentansprüchen sowie eine Schritt-für-Schritt-Darstellung des Prozesses, 
wie dieses Patent im aktuellen rechtlichen Rahmen angemeldet werden könnte, 
einschließlich der potenziellen Herausforderungen und Hürden. 
\\

Diese Arbeit zielt darauf ab, 
ein umfassendes Verständnis der rechtlichen 
und praktischen Aspekte der Patentierbarkeit von KI-generierten Computerprogrammen zu vermitteln 
und mögliche Lösungsansätze für die identifizierten Herausforderungen aufzuzeigen. 


\section{Umfang\label{sec:scope}}


Ziel dieser Arbeit ist es, 
die rechtlichen Rahmenbedingungen sowie die praktischen Herausforderungen der Patentierung 
von KI-generierten Computerprogrammen im deutschen Patentrecht zu analysieren. 
Dabei wird auf die Aspekte Neuheit, die Erfinderfrage, 
die erfinderische Tätigkeit und allgemeine Voraussetzungen für die Patentierbarkeit eingegangen 
und diese unter Berücksichtigung relevanter Paragraphen im deutschen Patentgesetz (PatG), 
sowie anderer relevanter Rechtsquellen ausgearbeitet. 
\\

Gemäß § 1 PatG müssen Erfindungen auf einer erfinderischen Tätigkeit beruhen. 
Die Bewertung der erfinderischen Tätigkeit bei KI-generierten Programmen ist ein weiterer 
zentraler Punkt dieser Arbeit. Hier wird untersucht, 
wie der kreative Beitrag einer KI im Vergleich zu menschlichen Erfindern bewertet wird und 
welche Standards im deutschen Patentrecht angesetzt werden, um die erfinderische Schöpfungshöhe zu bestimmen.
\\

Gemäß § 3 PatG müssen Erfindungen neu sein, um patentierbar zu sein. 
Hier wird untersucht, 
wie das deutsche Patentrecht die Neuheit von KI-generierten Softwarelösungen definiert und bewertet. 
Dies beinhaltet eine Analyse von § 3 PatG und relevanten Rechtsprechungen, um festzustellen, 
welche Anforderungen erfüllt sein müssen, 
damit eine KI-generierte Software als neu gilt und somit patentierbar ist. 
\\

Gemäß § 4 PatG muss eine Person als Erfinder genannt werden. 
Ein zentraler Aspekt dieser Arbeit wird die Frage sein, 
wer rechtlich als Erfinder einer KI-generierten Software gilt. 
Dabei wird analysiert, 
ob und in welchem Ausmaß eine Künstliche Intelligenz selbst als Erfinder anerkannt werden kann
oder ob diese Rolle dem menschlichen Entwickler oder dem Bediener der KI zufällt. 
Dies beinhaltet eine genaue Betrachtung von § 4 PatG und dazugehörigen Rechtsprechungen.
\\

Neben den spezifischen Bestimmungen werden die allgemeinen Voraussetzungen 
für die Patentierbarkeit von Computerprogrammen gemäß § 1 Abs. 3 PatG untersucht. 
Dies umfasst die Abgrenzung zu anderen geistigen Eigentumsrechten wie dem Urheberrecht, 
gemäß dem Urheberrechtsgesetz (UrhG), welches primär Schutz für schöpferische Werke bietet, 
sowie eine klare Differenzierung vom internationalen Patentrecht, 
das teilweise andere Anforderungen und Verfahrensweisen für die Patentierung von Software und Technologien hat.
\\

Abbildung \ref{fig:patentrecht} zeigt die Zusammenhänge zwischen Patentgesetzen, 
künstlicher Intelligenz und Computerprogrammen.
\\
\begin{figure}[htb]
  \centering
  \includegraphics[width=\textwidth]{img/Patentrecht Übersicht.pdf}\\
  \caption{Patentrecht Gesamtbild}\label{fig:patentrecht}
\end{figure}

\section{Gliederung\label{sec:outline}}


\textbf{Kapitel \ref{cha:chapter2}} 
Dieses Kapitel bietet einen umfassenden Überblick über die Grundlagen des deutschen Patentrechts, 
von künstlicher Intelligenz und Computerprogrammen. 
Es erläutert die zentralen Konzepte des Patentrechts, 
den Aufbau, 
die Funktionsweise und die verschiedenen Arten von KI und definiert den Begriff des Computerprogramms. 
Darüber hinaus werden relevante Werke aus der juristischen Fachliteratur vorgestellt, 
die als Grundlage für die folgenden Analysen dienen.
\\
\\
\textbf{Kapitel \ref{cha:chapter3}} 
In diesem Kapitel wird die Patentierbarkeit von Erfindungen untersucht,
die durch künstliche Intelligenz erstellt werden, 
sowie die Patentierbarkeit von Computerprogrammen im deutschen Patentrecht. 
Es befasst sich mit Anforderungen an die Neuheit 
und erfinderische Tätigkeit solcher Innovationen gemäß dem deutschen Patentgesetz. 
Besonderer Fokus liegt auf der Fragestellung, 
ob und wie KI als Erfinder rechtlich anerkannt werden kann 
und welche rechtlichen Herausforderungen dies mit sich bringt.
\\
\\
\textbf{Kapitel \ref{cha:chapter4}} 
Dieses Kapitel präsentiert ausgewählte Fallbeispiele aus der Rechtsprechung, 
die sich mit künstlicher Intelligenz und Computerprogrammen im Patentrecht beschäftigen. 
Es untersucht, wie Gerichte bisherige Fälle behandelt haben 
und welche Schlussfolgerungen daraus für die aktuelle Praxis gezogen werden können. 
Die Analyse konzentriert sich auf die Anwendung der Patentrechtsprinzipien auf technische Innovationen, 
die durch KI generiert werden, 
sowie auf die Herausforderungen bei der rechtlichen Einordnung solcher Technologien.
\\
\\
\textbf{Kapitel \ref{cha:chapter5}} 
Im Mittelpunkt dieses Kapitels steht die Entwicklung eines hypothetischen Patentantrags 
für eine KI-generierte Software. 
Es bietet eine detaillierte Beschreibung technischer Merkmale des Computerprogramms, 
formuliert Patentansprüche und skizziert den Prozess der Patentanmeldung im Rahmen des deutschen Rechtssystems. 
Potenzielle Herausforderungen und Lösungsansätze bei der Patentierung werden dargestellt, 
um Einblicke in den Anmeldeprozess zu geben.
\\
\\
\textbf{Kapitel \ref{cha:chapter6}} 
Dieses Kapitel zieht eine Bilanz der vorangegangenen Untersuchungen und Analysen. 
Es beleuchtet die ermittelten Ergebnisse in Bezug auf die rechtliche Bewertung 
von KI-generierten Computerprogrammen im deutschen Patentrecht. 
Dabei werden die wichtigsten Erkenntnisse herausgearbeitet 
und offene Fragen sowie potenzielle Weiterentwicklungen im Patentrecht dargestellt.
\\
\\
\textbf{Kapitel \ref{cha:chapter7}} 
Abschließend fasst das Kapitel Fazit und Ausblick die zentralen Ergebnisse zusammen 
und gibt einen Ausblick auf zukünftige Entwicklungen im Bereich der Patentierbarkeit 
von KI-generierten Computerprogrammen. 
Es hebt die Bedeutung der rechtlichen Klarstellungen hervor, 
die notwendig sind, um Innovationen im Bereich der künstlichen Intelligenz zu schützen, 
und stellt mögliche Ansätze für eine Weiterentwicklung des deutschen Patentrechts dar.