\chapter{Auswertung\label{cha:chapter6}}

\section{Darstellung der Ergebnisse}
Das Urteil der KI "DABUS" ist wegweisend für die Patentierbarkeit von
KI-Erfindungen. In Deutschland ist per BGH-Beschluss 
festgelegt, dass eine KI
zur Erstellung von Erfindungen verwendet werden darf. Sie
darf auch den größeren Anteil an der Erstellung des Patents
haben. Ein menschlicher Anteil ist allein durch die 
fehlende Anwesenheit von komplett autonom agierenden KIs gegeben.
Der Erfinder ist die menschliche Person (oder Personen),
welche mit der KI interagiert um die Erfindung zu erstellen.
Damit sind die formellen Anforderungen bei der Anmeldung
von durch KI geschaffener Computerprogramme im deutschen
Patentrecht definiert, da sich die restlichen formellen
Anforderungen nicht von denen 
ohne KI oder ohne Computerprogrammen unterscheiden.
Die materiellen Herausforderungen bei der Patentierbarkeit
von Computerprogrammen sind vor allem die Frage der Technizität.
Da eine Erfindung auf einem Gebiet der Technik liegen muss
um patentierbar zu sein, ist es wichtig zu klären, ab wann
Computerprogramme als technisch gelten.
Diese Computerprogramme werden dann nicht von der Patentierbarkeit
ausgeschlossen. Sie sind keine Computerprogramme "als solche",
sondern welche mit technischem Charakter.
Über verschiedene Rechtssprechungen vom Dispositionsprogramm, über
das Antiblockiersystem und die Entscheidung zum Tauchcomputer sowie 
vielen weiteren, ist in Deutschland eine Rechtsgrundlage geschaffen worden
um die Technizität von Computerprogrammen zu bewerten.
Das DPMA fasst die Rechtssprechungen in einen drei Schritte Plan 
zur Beurteilung des technischen 
Charakters von computerimplementierten Erfindungen zusammen.
Eine Erfindung muss auf einem Gebiet der Technik liegen,
ein technisches Problem lösen und erfinderische Tätigkeit und Neuheit 
aufweisen. Wobei nur Aspekte berücksichtigt werden,
die die Lösung des technischen Problems beeinflussen für die
erfinderische Tätigkeit und Neuheit.
Damit lässt sich die Beurteilung der Technizität
von Computerprogrammen in Deutschland durchführen.
Wenn es um die Beurteilung des Neuheit von durch KI generierten 
Erfindungen geht, ist es wichtig zu beachten, dass eine KI 
durchaus neuartige Erfindungen hervorbringen kann.
Eine KI lernt auf historischen Daten, trotzdem ist
eine neue Kombination aus bereits bekannten Technologien, 
neuartig, wenn es sie in der bisherigen Form noch nicht gab. 
Die Frage ist dann, ob diese Form dann nicht nahliegend ist.
Eine erfinderische Tätigkeit kann trotz geringem 
zutun des menschlichen Nutzers gegeben sein, wenn 
die Nutzung für die Entwicklung der Erfindung für einen Fachmann 
im Fall von Verfahrenspatenten
nicht naheliegend ist 
oder die Erfindung an sich im Fall von Erzeugnispatenten 
nicht naheliegend ist.

Im Falle des hypothetischen Patentantrages \ref{cha:chapter5}
wurde eine Erfindung generiert, welche durch die KI ChatGPT-4o
erstellt wurde.
Die Erfindung erfüllt die nötigen formellen Vorraussetungen
unter Angabe des Nutzers der KI als Erfinder. In dem Fall
wird die KI als Werkzeug deklariert.
Da die generierte Erfindung "Intelligentes Energiemanage-
mentsystem für Internet of Things (IoT)-basierte Haushalte"
Schutzansprüche für mehrere Computerprogramme beinhaltet, 
wird der technische Charakter nach den Maßstäben des DPMA geprüft.
In diesem Fall weisen die Schutzansprüche mit 
Bezug auf Computerprogramme, bzw. Verfahren,
welche ein Computerprogramm darstellen, 
einen technischen Charakter auf und
sind keine Computerprogramme "als solche".
Die Neuheit ist bei dem erstellten Patent ebenfalls gegeben,
da die konkrete Kombination, wie sie in den Schutzansprüchen
dargestellt wurde, noch nicht Stand der Technik ist.
Jedoch bleibt die Frage der Erfinderischen Höhe bei der
Erfindung bestehen.
Dadurch, dass die Schutzansprüche, welche von der KI
erstellt wurden, den Einsatz von maschinellen
Lernen im Bereich von IoT behandeln und dies
für einen Fachmnn in diesem Bereich naheliegend
ist, ist ein Verfahrenspatent dafür nicht mit einer
erfinderischen Tätigkeit verbunden.
Außerdem bieten die weiteren Ansprüche mit Bezug auf 
Erzeugnispatente ebenfalls keine
für einen Fachmann nicht naheliegenden Innovationen,
weshalb die erfinderische Höhe bei allen Schutzansprüchen fehlt.
In Sachen erfinderische Tätigkeit 
fehlt es ChatGPT-4o an der nötigen Kreativität bei 
der von im hypothetischen Patentantrag generierten Erfindung.
Einzig und Alleine der Einsatz von ChatGPT zum Erstellen
von patentierbaren Erfindungen könnte im Bereich Iot 
als erfinderisch gelten, wird hier aber nicht als Anspruch
formuliert und geprüft.
Das hypothetische Patent ist trotz der im Prompt 
verwendeteten Anforderung von Patentierbarkeit,
nicht patentierbar. 
\section{Analyse}

Die Patentierbarkeit von durch KI 
geschaffener Computerprogramme ist ein sehr aktuelles Thema 
und es ist durchaus möglich, dass eine KI mit simplen
Inputs ein patentierbares Computerprogramm erzeugen kann.
Die formellen Vorrausetzungen der Patentanmeldung, 
dass ein Nutzer der KI als
Erinder eintritt, stellen in Deutschland keine große Hürde
dar, da die Wahrheitspflicht laut BGH hierbei
nicht verletzt wird. Die Technizität eines von einer
KI erstellten Programmes 
kann gewährleistet werden, 
indem der KI mitgegeben wird, dass das 
Computerprogramm ein technisches Problem lösen 
soll oder generell, dass es patentierbar sein soll.
Es gibt mittlerweile dutzende Fälle in denen 
Computerprogramme mit technischen Charakter patentiert wurden
und auch die oben aufgeführten Programme lösen
ein technisches Problem mit technischen Mittel und
sind durchaus patentierbar. Bei den materiellen
Vorrausetzungen der Patentanmeldung ist es Einzelfall
abhängig, ob die Erfindung genug erfinderische Tätigkeit
und Neuheit aufweist. In dem hypothetischen
Patentantrag wurde gezeigt, dass die Erfindung
schnell am Punkt der erfinderischen Tätigkeit scheitern
kann, da eine KI auf den ersten Blick kreativ erescheinde
Erfindungen liefert, diese jedoch für einen Fachmann
als sehr naheliegend angesehen werden können.
Dies kann umgangen werden indem ein Ingenieur der 
Branche indem die KI eine Erfindung generiert die 
Erfindung prüft, gegebenenfalls anpasst oder bei
mehreren die naheliegenden aussortiert. Dabei ist 
dann jedoch auch schon ein höherer menschlicher 
Aufwand mit verbunden und es stellt sich die Frage,
ob die Erfindung noch als KI generiert angesehen werden
kann, wenn es sich im ein Zusammenspiel von Mensch
und Erfindung handelt. Es ist dem BGH zuzustimmen, 
dass derzeit noch keine KI alleine etwas erfinden kann,
ohne den Input eines Menschen und es stellt sich
auch die Frage, ob eine KI ohne einen Ingenieur
oder fachkundigen in der Branche eine patentierbare
Erfindung erschaffen kann.
\section{Interpretation}
Die Analyse der Gesetzgebung und des hypothetischen Patentantrages
haben gezeigt, dass Erfindungen von KI durchaus das Potential
haben, in nächster Zeit patentierbare Computerprogramme zu
generieren. Vorallem der hypothetische Patentantrag hat gezeigt,
wie einfach es ist eine Erfindung mit einer KI, in diesem Fall
ein LLM zu generieren. Wenn eine spezialisiertere KI verwendet 
wird und diese von einem Fachmann bedient wird ist es
möglich mit wenig Aufwand viele patentierbare Erfindungen zu 
generieren. Diese werden so lange patentierbar bleiben,
bis es als naheliegend angesehen wird die verwendete KI
als Werkzeug in der Branche zu verwenden oder sich
die Rechtssprechung ändert. Die Entwicklung im 
Bereich von KI ist sehr schnell, so dass es durchaus vorstellbar
ist, dass eine Art Erfinder-KI entwickelt wird, die schaut,
wie der aktuelle Stand der Technik ist und wie sie etwas nicht 
naheliegendes erfinden kann. Die Erfindungen wären 
dann auf Relevanz und gewerbliche Anwendbarkeit zu prüfen, 
von einem Menschen oder ebenfalls von einer darauf 
trainierten KI. 