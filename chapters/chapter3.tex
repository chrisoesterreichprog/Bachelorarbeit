\chapter{Patentierbarkeit von KI generierten Erfindungen und Computerprogrammen \label{cha:chapter3}}


\section{Einführung in das Patentrecht\label{sec:Patentrecht}}


Das \gls{DPMA} ist die zentrale Behörde in Deutschland, 
die für den Schutz von geistigem Eigentum zuständig ist und 
beschreibt den Nutzen von Patenten wie folgt:
\\
\\
\textit{''Mit Patenten können Sie Ihre technischen Erfindungen 
(innovative Produkte oder Verfahren) vor unerwünschter Nachahmung schützen. 
Patente belohnen ihren Inhaber 
oder ihre Inhaberin durch ein befristetes und räumlich begrenztes Nutzungsmonopol.''} 
\cite{DPMAPatentschutz}
\\
\\
Das Bundespatentgericht (BPatG) ist für Streitigkeiten über Patente zuständig, 
insbesondere für Nichtigkeitsklagen und Beschwerden gegen Entscheidungen des DPMA.

Das deutsche Patentgesetz(PatG) \cite{PatGNichtamtlichesInhaltsverzeichnis} 
regelt die rechtlichen Rahmenbedingungen für Patente in Deutschland.
Das Patentrecht in Deutschland ist ein spezieller Teil des gewerblichen Rechtsschutzes, 
der wiederum zum Bereich des Immaterialgüterrechts gehört. 
Es ist durch das Grundgesetz geschützt, insbesondere durch Art. 14 GG, 
der das Eigentum und das Erbrecht gewährleistet. 
Im materiellen Sinne gehören Patente zum Eigentum.
Ein weiterer Teil des Immaterialgüterrechts ist das Urheberrecht. 
Dieses grenzt sich von anderen Teilen des
gewerblichen Rechtsschutzes wie das Markenrecht und Designrecht dadurch ab, das 
es kreative Leistungen schützt\cite{GewerblicherRechtschutzUnd}. 


Ein Patent kann gemäß §49 Abs.1 PatG von der Prüfstelle des DPMA erteilt werden. 
Um die Patentierbarkeit von durch künstliche Intelligenz geschaffener Computerprogramme 
zu prüfen fokussiert sich diese Arbeit auf die Erteilung von Patenten. 
Die Erteilung von Patenten lässt sich in formelle und materielle Vorrraussetzungen 
gliedern, was sich aus §49 Abs.1 PatG ableitet. 
Das Deutsche Patent- und Markenamt prüft auf Antrag (§44 Abs.1 PatG) 
die formellen und materiellen Vorrausetzungen.

\section{Formelle Vorraussetzungen}

Zu den formellen Vorraussetzungen gehören:
\begin{enumerate}
    \item Anmeldung und Form, §§ 34, 37 und 38 PatG 
    \vspace{-0.11in} 
    \item Beseitigung gerügter Mängel, § 45 Abs. 1 PatG
\end{enumerate}

Patente müssen angemeldet werden (§ 34 Abs. 1 PatG). 
Gemäß § 34 Abs. 3 PatG muss die Anmeldung den Namen der/des Anmelders*in 
(Nr. 1), einen Antrag auf Erteilung des Patents, 
in dem die Erfindung kurz und genau bezeichnet ist (Nr. 2), 
einen oder mehrere Patentansprüche (Nr. 3), 
eine Beschreibung der Erfindung (Nr. 4) sowie die Zeichnungen, 
auf die sich die Patentansprüche oder die Beschreibung beziehen (Nr. 5), 
enthalten. 
Außerdem muss die Erfindung vollständig und deutlich offenbart sein (§ 34 Abs. 4 PatG) 
und nur eine einzige Erfindung enthalten (§ 34 Abs. 5 PatG). 
Paragraph 37 des PatG befasst sich mit der korrekten Erfinderbennenung 
und Paragraph 38 mit Änderungen der Anmeldung. 
\\

Sind die oben genannten formellen Anforderungen nicht erfüllt § 45 Abs. 1 PatG, 
wird der Anmelder aufgefordert, diese innerhalb einer bestimmten Frist zu beseitigen.

Wenn die gerügten Mängel beseitigt wurden 
oder es gemäß §§ 34, 37 und 38 PatG keine gerügten Mängel gibt 
sind die formellen Vorrausetzungen für die Erteilung eines Patents erfüllt.

Bei der Patentierbarkeit von KI generierten Erfindungen ist ein besonderes Augenmerk
auf § 34 Abs. 3 PatG Nr.1 zu werfen, da hier keine natürliche Person
die Erfindung hergestellt hat sondern eine KI.

\subsection{KI als Erfinder}
Die Bennung der KI als Erfinder ist ziemlich naheliegend, 
da in § 124 PatG zur Vollständigigkeit und Wahrheit vor dem
Deutschen Patent- und Markenamt, 
dem Patentgericht und dem Bundesgerichtshof aufgerufen wird.
\\
Am 17.10.2018 ging beim \gls{EPA} ein Patent ein, 
welches eine leere Zeile als Erfinder aufwies und später mit 
dem Namen der KI 
"DABUS"(Device for the Autonomous Bootstrapping of Unified Sentience) 
als Erfinder ergänzt wurde. 
Das EPA hat daraufhin entschieden, 
dass diese Bennenung nicht dem Artikel 81 Abs.19 (1) \gls{EPÜ}  genügt,
da der Erfinder eine natürliche Person sein muss, sowie ein 
Familienname, Vorname und eine Adresse angegeben werden muss.
Damit sind die formellen Vorrausetzungen an das Patent mit dem Aktenzeichen
EP 18 275 163 nicht erfüllt und das Patent wurde nicht erteilt. 
Außerdem wurde entschieden, das Namen die Dingen gegeben werden nicht
mit Namen natürlicher Personen gleichzusetzen sind. Maschinen oder KI
Systeme haben keine Rechte, die durch den Namen ausgeübt werden können 
\cite{EPA27012020}. 
In einem Beschluss vom 20.Oktober 2020 wird das Thema im Europäischem
Parlament nochmals aufgegriffen und erkannt, 
dass die aktuelle Gesetzeslage nur Erfindungen berücksichtigt,
die von Menschen mit Hilfe von KI geschaffenen wurden. 
Eine klare Unterscheidung muss getroffen werden um diese 
von vollständig autonom von KI geschaffenen Erfindungen abzugrenzen
\cite{TextsAdoptedIntellectual}.
Eine Studie, welche von der EU Kommission beauftragt wurde 
sieht die derzeitige Gesetzeslage dahingegen als ausreichend 
und fordert erst Handlungsbedarf bei dem Einsatz von starken KIs.
Es wird vorgeschlagen dann ein spezifisches Roboterrecht zu schaffen, 
um den Umgang mit intelligenten Maschinen zu regeln 
und die Informationssicherheit zu gewährleisten
\cite{gutaAPPLICABILITYGDPRARTIFICIAL2022}.
Jedoch ist es auch heute schon möglich teilweise 
Unabhängigkeit von menschlicher Einwirkung
zu schaffen, da auch schwache KIs durch Deep Learning 
Fähigkeiten nach einwirken durch den Menschen erlangen, wie bei DABUS
\cite{surdenMachineLearningLaw}\cite{dornisDornisSchopfungOhne2021}.
Eine weitere Form autonomer KIs sind die Genetic Breeding Algorithmen,
welche auch einen minimalen menschlichen Input beim Erschaffungsprozess 
verlangen.
\\
Bei der Anmeldung vor dem DPMA am 17. Oktober 2019 mit dem Aktenzeichen
10 2019 129 136 kann die künstliche Intelligenz 
DABUS ebenfalls nicht als Erfinder in Kraft treten. 
Eine künstlichen Intelligenz, 
erfüllt nicht die gesetzlichen Anforderungen an die Erfinderbenennung 
gemäß § 37 PatG und § 7 PatV. In § 7 PatV wird ebenfalls
Familienname, Vorname und eine Adresse gefordert. 
Die KI "DABUS" wird vom DPMA ebenfalls als 
Sache bzw. Machine angesehen, welche kein Träger von Rechten sein kann.
Wenn der Anmelder der einzige ist, 
der die Maschine genutzt hat, 
und keine andere Person zur Erfindung beigetragen hat, 
kann er sich selbst als Erfinder benennen. 
Falls der Anmelder Bedenken hat, 
die Nutzung der künstlichen Intelligenz zu verschweigen, 
kann er die Nutzung in der Beschreibung der Patentanmeldung angeben 
und so der Wahrheitspflicht gemäß § 124 PatG gerecht werden.
\cite{BPatG21122021}.
Einen weiteren Ansatz schlagen Konertz und Schönhof mit einem 
„erfinderloses Patent“ vor, das es der Person, 
die das Computersystem nutzt, erlaubt, 
die Rechte an der durch die Maschine generierten Erfindung zu beanspruchen
\cite{konertzErfindungenDurchComputer2018}.

\subsection{De lege lata}
Nach geltendem Recht "de lege lata" ist es in Deutschland und Europa
derzeit nicht möglich eine KI als Erfinder in Erscheinung treten zu lassen.
Die Gesetzeslage sieht vor, dass ein Erfinder eine natürliche Person sein muss
dabei ist es egal, ob eine "schwache KI" oder eine "starke KI" verwendet wurde.
Als Erfinder tritt dann der Nutzer der KI ein, welcher diese dann offiziell
als Werkzeug benutzt hat um eine technische Erfindung zu produzieren.

\subsection{De lege ferenda}
Nach zukünfigem Recht könnte sich einiges tun, die Debatte wird von verschiedenen 
Instanzen wie dem europäischen Parlament immer wieder aufgenommen und es 
werden in Zukunft weitere Fälle von Erfindungen folgen die eine klare rechtliche 
Klärung bedürfen. 
Bei den bisherigen schwachen KIs ist es noch möglich die KI als Werkzeug
zu sehen, wobei es dort schon Schwierigkeiten geben könnte, wenn eine KI,
welche durch einen Genetic Breeding Algorithmus erstellt würde eine Erfindung
schafft, ohne vorher einen Input bekommen zu haben.
Ab dem Punkt, wo kein Mensch mehr im Erfindungsprozess beteiligt,
sondern nur beim Entwicklungsprozess der KI beteiligt ist,
braucht es auf jeden Fall ein zusätzliches Recht, wie das 
oben erwähnte spezifische Roboterrecht.
Starken KIs werden spätestens, eine klare rechtliche Abgrenzung 
brauchen, 
da ab diesem Punkt Erfindungen komplett autonom von einer KI erschfaffen werden.
Dann stellt sich die Frage ob die KI doch als Erfinder auftreten kann 
und eigene Rechte besitzen darf. Ähnlich wie in dem Entwurf des Europäischen
Parlaments vom 31.5.2016 mit Empfehlungen an die Kommission zu zivilrechtlichen Regelungen im
Bereich Robotik \cite{delvauxMitEmpfehlungenKommission}. Dieser schlägt 
eine Einführung eines eigenen Rechtsstatus 
für Roboter als "elektronische Personen" vor die Rechte und Pflichten zu haben, 
ähnlich wie Menschen.
\\
\subsection{Exkurs: Internationales Patentrecht\label{sec:intp}}
\todo{exkurs1}
\section{Materielle Vorrausetzungen}

Die materiellen Vorraussetungen der Patentanmeldung sind 
in den Paragraphen §§ 1 – 5 PatG geregelt 
und lassen sich unterteilen in folgende Punkte:

\begin{enumerate}
    \item Erfindung auf einem Gebiet der Technik, § 1 Abs. 1 PatG
    \begin{enumerate}
    \vspace{-0.05in}
    \item Ausschluss, §§ Art. 1 Abs. 3 PatG und § 1 Abs. 4 PatG
    \end{enumerate}
    \vspace{-0.11in} 
    \item Neuheit, § 1 Abs. 1 i.V.m. § 3 PatG
    \vspace{-0.11in} 
    \item Erfinderische Tätigkeit, § 1 Abs. 1 i.V.m. § 4 PatG
    \vspace{-0.11in} 
    \item Gewerbliche Anwendbarkeit, § 1 Abs. 1 i.V.m. § 5 PatG
    \vspace{-0.11in} 
    \item Ausschluss, § 2 PatG
\end{enumerate}

\subsection{Technische Erfindung}

Im ersten Absatz des ersten Paragraphen im Patentgesetz wird festgelegt,
dass eine Erfindung auf einem Gebiet der Technik liegen muss, neu sein und
gewerblich anwendbar.
Eine technische Erfindung liegt laut Haedicke dann vor, 
wenn die Erfindung aus dem Bereich der Physik, 
Chemie oder den Ingenieurswissenschaften ist, 
welche sog. Gebiete der Technik darstellen.
In den letzen Jahrzehnten hat sich der Begriff 
"Technik" auch auf Erfindundungen der Biotechnologie, 
Telekommunikations- und Computertechnologie ausgeweitet 
\cite{haedickeEinfuhrung2020}
Der \gls{BGH} erstellte die
sog. „Rote-Taube“-Formel, welche technisch als  
„eine Lehre zum planmäßigen Handeln 
unter Einsatz beherrschbarer Naturkräfte zur Erreichung eines 
kausal übersehbaren Erfolgs definiert.“\cite{BGH27031969}  
Dabei genügt es „wenn die beanspruchte Lehre den Einsatz technischer Geräte umfasst“
\cite{BGH3020152015}\cite{BGH2420112011a}. 
Aufgrund der ständigen Entwicklung lässt sich 
jedoch der Begriff der „technischen Erfindung“ 
nicht abschließend definieren \cite{haedickeEinfuhrung2020}.
\\
In Paragraph 1 Abs. 3 PatG werden Gegenstände und Tätigkeiten festgelegt, 
die nicht als Erfindung angesehen werden dürfen. 
Diese sind nicht patentfähig als solche (§ 1 Abs. 4 PatG). 
Ausgeschlossene Erfindungen sind Entdeckungen, 
sowie wissenschaftliche Theorien und mathematische Methoden, 
ästhetische Formschöpfungen, Pläne, Regeln und Verfahren für gedankliche Tätigkeiten, 
für Spiele 
oder für geschäftliche Tätigkeiten sowie Programme für Datenverarbeitungsanlagen, 
sowie die Wiedergabe von Informationen (§. 1 Abs. 3 PatG). 
\\

\subsubsection{Patentierbarkeit von Computerprogrammen}
Der Punkt Programme für Datenverarbeitungsanlagen sind grundsätzlich
von der Patentierbarkeit ausgeschlossen stellt Schwierigkeiten in der 
Patentierbarkeit von durch künstliche
Intelligenz geschaffener Computerprogramme dar. 
Jedoch betrifft der Ausschluss nur Programme "als solche", 
was bedeutet, 
dass Computerprogramme in bestimmten Zusammenhängen patentierbar sind, 
wenn sie eine technische Aufgabe lösen und technische Merkmale aufweisen.
Beispiele für patentierbare Software sind technische Anwendungsprogramme, 
die Messergebnisse verarbeiten, 
technische Einrichtungen überwachen oder in technische Systeme eingreifen
\cite{RedekerITRechtSchutz}.
Bevor Beispiele von patentierbaren Computerprogrammen folgen, ist es
nötig den Begriff des Computerprogrammes eindeutig zu definieren.
Ein Computerprogrammm ist laut ISO/IEC 2382-1:1993 
eine Kombination von Anweisungen und Deklarationen in einer
Programmiersprache, die einen Computer dazu bringen, 
Funktionen zur Lösung eines Problems auszuführen 
\cite{instituteofelectricalandelectronicsengineersinc.ISO47652010}.
Das in der Programmiersprache geschriebene Programm(Quellprogramm) wird 
mittels eines Sprachcompilers in ein in Maschinensprache geschriebenes 
Programm(Objektprogramm) aus Nullen und Einsen umgewandelt \cite{WasIstProgramm}.
Computerprogramme werden nach dem Wirtschaftslexikon Gabler in Systemprogramme
und Anwendungsprogramme unterteilt. 
Ein Anwendungsprogramm löst dabei eine bestimmte Aufgabe des Anwenders, 
wie z.B. Ticketreservierungen oder Parkraumüberwachung 
\cite{lackesDefinitionAnwendungsprogramm}. 
Während ein Systemprogramm ein Bestandteil des Betriebssystems ist und 
für den Nutzer nicht sichtbare Teile der internen Steuerung des Computer übernimmt,
wie die Orchestrierung der als nächstes zu bearbeitenden Aufgabe \cite{lackesDefinitionSystemprogramm}.
Für beide Arten von Programmen gibt es mögliche Ausprägungen, 
welche patentierbar sein können, so kann ein mögliches Szenario bei einem Anwendungsprogramm sein,
dass ein Algorithmus entwickelt wurde der in der Branche noch nicht vorhanden ist. Z.B. ein
Bildverarbeitungsprogramm, das einen neuen Algorithmus verwendet, um Bilder zu verbessern.
Oder im Falle von Systemprogrammen ein Betriebssystem mit einem 
innovativen Verfahren zur Erkennung und Verhinderung von Malware.
\paragraph{Abgrenzung zur Software}
Ein Begriff der heutzutage oft synonym zu dem Begriff Computerprogramm benutzt wird,
ist Software. Software ist laut ISO/IEC 2382-1:1993 eine Sammlung von Computerprogrammen, 
Daten und Bibliotheken und somit ist ein Computerprogramm nur ein Bestandteil einer Software.
Software verwendet Computerprogramme als Tools um individuelle Anweisungen auszuführen 
\cite{ComputerProgrammeUnverzichtbareComputerprogramme}.
Software wird nach dem oben genannten ISO-Standard außerdem in 
Systemsoftware, Unterstützungssoftware und Anwendungssoftware unterteilt.
Systemsoftware ist Software, welche die Hardware des Computers steuert, 
wie z.B. Betriebssysteme oder Gerätetreiber. Unter Unterstützungssoftware 
fällt Software, die die Entwicklung und Ausführung von Anwendungssoftware
unterstützt, wie z.B. Compiler oder Texteditoren. Anwendungssoftware ist
Software, die für die Lösung von Problemen oder die Durchführung von Aufgaben
entwickelt wurden, wie z.B. Textverarbeitungsprogramme oder Spiele
\cite{instituteofelectricalandelectronicsengineersinc.ISO47652010}.
\\
\paragraph{Urteile mit Bezug auf Patentierbarkeit von Computerprogrammen}
Es gibt viele Präzendenzfälle in den Computerprogramme patentiert worden sind.
Das Europäische Patentamt erteilt außerdem Patente auch für technische Umsetzungen.
Auch wenn eine mathematische Methode keine direkte technische Anwendung hat, 
kann sie patentierbar sein, 
wenn sie speziell für eine technische Umsetzung angepasst wurde. 
Beispiele sind Optimierungen für Hardware-Architekturen, 
wie die Nutzung von GPUs für maschinelles Lernen \cite{MathematischeMethoden}. 
Eingehend beschäftigt mit der Thematik haben sich bereits einige Juristen,
wie z.B. 
Hon. Prof. Dr. iur. Klaus-J. Melullis
(Leiter der Forschungsgruppe Patentrecht am Karlsruher Institut für Technologie) und 
Dr. Matthias Koch (Rechtsanwalt beim BGH)\cite{melullisEPUArt522023} im Zusammenhang 
mit dem Europäischen Patentübereinkommen 
oder Prof. Dr. jur. Dr. rer. pol. Jürgen Ensthaler 
(Lehrstuhlinhaber für Wirtschafts-, Unternehmens - und Technikrecht an der TU Berlin)
im Zusammenhang mit dem deutschen PatG.
\cite{ensthalerEnsthalerBegrenzungPatentierung2013}.
Als Beispiel für ein Computerprogrammpatent auf deutscher 
Ebene dient die Analyse 
und Steuerung eines Flugzeugzustands.
Das hier genutzte Computerprogramm benutzt ein Verfahren, 
das anhand von Messwerten Erkenntnisse 
über den Zustand eines Flugzeugs gewinnt und 
die Funktionsweise eines Systems beeinflusst 
\cite{BGH3020152015}.
Hier liegt eine technische Problemlösung vor, 
da die Methode zur Steuerung eines technischen Systems eingesetzt wird.
Das Europäische Patentamt hat mit dem Patent EP0005954 
„Verfahren und Vorrichtung zur verbesserten digitalen Bildverarbeitung“ 
einen entscheidenden Meilenstein gesetzt, 
der den Weg für die Patentierbarkeit mathematischer Methoden und 
Computerprogramme geebnet hat.
Das hier aufgeführte Computerprogramm ist ein 
Verfahren zur digitalen Bildverarbeitung \cite{EPThisFile}.
Der technische Charakter liegt in der Verbesserung 
der Bildqualität durch ein spezielles Filterverfahren.
\\
Die bloße Implementierung einer Rechenmethode auf einem Computer 
verleiht ihr noch keinen technischen Charakter. 
Entscheidend ist die konkrete Anwendung der Methode 
in einem technischen Zusammenhang, der einen unmittelbaren Effekt in der 
physischen Welt erzeugt \cite{melullisEPUArt522023}.
So sind mehrere Patente abgelehnt worden,
wie, Verfahrenen, 
die lediglich der Auswertung von Daten auf statistischer Basis dienen.
Dies war der Fall bei den Beschlüssen 17 W (pat) 74/07\cite{BPatG10012012}
und 17 W (pat) 6/00 \cite{BPatG01032001} vom Bundespatentgericht.
Das EPA sieht Verfahren zum Sammeln und Auswerten von Daten 
im Rahmen von betriebswirtschaftlichen Prozessen als nicht patentfähig an 
(siehe T 154/04 \cite{EuropaischesPatentamt152006}),
da sie kein technisches Problem lösen.
\\
Nach dieser Definition von Computerprogrammen und der Klärung von dem Begriff
"Technik" kann nun eine Abschätzung getroffen werden, 
ab wann Computerprogramme patentierbar sind. 
Durch die verschiedenen Urteile wird außerdem sichtbar 
ab wann ein Computerprogramm als technisch angesehen wird.
Jedoch bleibt die Patentierbarkeit von Computerprogrammen 
ein komplexes und umstrittenes Thema 
und internationale Standards könnten 
langfristig mehr Klarheit und Konsistenz schaffen.
Viele Computerprogrammpatente wurden erst abgelehnt und erst nach 
einer Beschwerde und erneuter Prüfung erteilt.
Wege um das Patentgesetz dahingehend zu vereinfachen schlägt 
Prof. Dr. jur. Dr. rer. pol. Jürgen Ensthaler vor. 
Eine mögliche Lösung wäre, die „als solche“-Formel durch eine Regelung zu ersetzen, 
wie sie für Gensequenzen in § 1a PatG besteht. 
Demnach könnten Algorithmen nur dann patentiert werden, 
wenn die konkrete technische Funktion klar benannt 
und in den Patentanspruch aufgenommen wird. 
Diese Funktionsbegrenzung würde verhindern, 
dass abstrakte mathematische Lehren, 
die für viele Anwendungen nutzbar sind, patentiert werden 
\cite{ensthalerEnsthalerBegrenzungPatentierung2013}. 
\subsection{Exkurs: Urheberrecht\label{sec:urh}}
\todo{exkurs2}






\subsection{Neuheit}
Eine Erfindung gilt als neu, wenn sie nicht zum Stand der Technik gehört. 
Der Stand der Technik umfasst alle Kenntnisse, 
die vor dem für den Zeitrang der Anmeldung
maßgeblichen Tag durch schriftliche 
oder mündliche Beschreibung, durch Benutzung oder in
sonstiger Weise der Öffentlichkeit zugänglich gemacht worden sind (§ 3 Abs. 1 PatG).
Als Stand der Technik gilt auch der Inhalt nationaler Patentanmeldungen 
in der beim Deutschen
Patentamt ursprünglich eingereichten Fassung mit älterem Zeitrang, 
die erst an oder nach dem für
den Zeitrang der jüngeren Anmeldung maßgeblichen Tag der Öffentlichkeit 
zugänglich gemacht
worden sind (§ 3 Abs. 2 Nr. 1 PatG). 
Bei KI-generierten Erfindungen kann es schwierig sein, 
den Stand der Technik umfassend zu bestimmen. 
KI-Modelle können auf umfangreiche Daten zugreifen und Lösungen generieren, 
die in kleinen Teilen bereits veröffentlicht, 
aber in dieser spezifischen Kombination noch nicht dokumentiert sind.
Im Kontext von KI stellt sich hier die Frage 
ob eine Erfindung überhaupt als neuartig angesehen werden kann, 
da KI basierte Erfindungen meist aus der Analyse großer Datenmengen bestehen
und der Generierung von Output aus diesen. 
Es werden nur bestehende Muster im Stand der Technik kombiniert 
und so nur durch Kombination aus anderen konkreten technischen Lösungen, 
welche bereits in irgendeiner Form öffentlich zugänglich waren ein neue generiert.
Eine Kombination bekannter technischer Lösungen kann als „neu“ angesehen werden,
wenn genau diese spezifische Kombination zuvor nicht offengelegt wurde, 
eine neu Ordnung von Informationen stellt jedoch keine Neuheit dar.
Dabei stellt sich jedoch die Frage ob die Kombination naheliegend und
somit keine erfinderische Höhe aufweist.

\subsection{Erfinderische Tätigkeit}
Eine Erfindung gilt als auf einer erfinderischen Tätigkeit beruhend, 
wenn sie sich für den Fachmann
nicht in naheliegender Weise aus dem Stand der Technik ergibt (§ 4 S. 1 PatG). 
Gehören zum Stand
der Technik auch Unterlagen im Sinne des § 3 Abs. 2 PatG, 
so werden diese bei der Beurteilung der
erfinderischen Tätigkeit nicht in Betracht gezogen (§ 4 S. 2 PatG).
In vielen Fällen agiert die KI als eine Art „Black Box“, 
bei der die Inputs vom Menschen bereitgestellt werden, 
die endgültigen Outputs jedoch nicht vollständig nachvollziehbar sind 
\cite{pauliniKIgenerierteErfindungPatentrechtliche}. 
Um die erfinderische Tätigkeit sinnvoll beurteilen zu können
muss klar sein, wie die KI zu ihrem Ergebnis kam. 
Wenn der Output für den Fachmann nicht aus dem bereits bekannten 
ableitbar ist, ist eigentlich eine erfinderische Tätigkeit gegeben.
Jedoch ist die erfinderische Tätigkeit so leicht zu reproduzieren,
dass eine erfinderische Höhe als fraglich angesehen werden kann.
Somit stellt sich die Frage: 

Ist nun auch dem Stand der Technik zugehörig alles was durch eine KI aus dem 
Stand der Technik hergeleitet werden kann? 

Dr. Joel Nägerl, Dr. Benedikt Neuburger und Dr. Frank Steinbach beschäftigen
sich mit dieser Thematik und kommen zu dem Schluss,
dass der Fachmann, der Beurteilungen vornimmt, 
in Zukunft KI-gestützte Systeme als „Lesebrille“ verwenden könnte.
Dies könnte dazu führen, dass Erfindungen, 
die heute noch mit einer erfinderischen Tätigkeit bewertet werden, 
in der Zukunft als naheliegend betrachtet und somit nicht patentierbar wären.
Dabei ist jedoch entscheidend, 
dass nur solche KI-Systeme herangezogen werden dürfen, 
die zum Zeitpunkt der Patentanmeldung verfügbar und gängig sind. 
Es ist unzulässig, den Bewertungsmaßstab durch die nachträgliche Nutzung 
einer später entwickelten, leistungsstärkeren KI zu erhöhen 
\cite{nagerlKunstlicheIntelligenzParadigmenwechsel2019}.

Insgesamt ist die Bewertung der erfinderischen Tätigkeit
von KI-generierten Erfindungen derzeit stark von der juristischen Auslegung abhängig. 
Solange der Beitrag der KI nicht klar und nachvollziehbar ist, 
bleibt es schwierig, 
die erfinderische Tätigkeit von KI-Erfindungen 
im patentrechtlichen Sinne zu bewerten.
Die Gesetzeslage könnte sich dem entsprend weiterentwickeln, 
die Reichweite 
vom Stand der Technik auf den Entwicklungshorizont von KI auszuweiten.
Dies ist bei dem derzeitigen Stand von KI durchaus sinnvoll,
jedoch ab der ersten starken KI vollkommen hinfällig.
\\
\subsection{Gewerbliche Anwendbarkeit}
Eine Erfindung gilt als gewerblich anwendbar, 
wenn ihr Gegenstand auf irgendeinem gewerblichen
Gebiet einschließlich der Landwirtschaft hergestellt 
oder benutzt werden kann (§ 5 PatG).
Das Erfindungen, 
welche mit/durch KI entstanden sind gewerblich anwendbar sind ist durchaus möglich.
KI-Systeme wie AlphaFold von DeepMind haben Fortschritte 
in der Medikamentenforschung ermöglicht, 
indem sie die 3D-Struktur von Proteinen präzise vorhersagen \cite{AlphaFold2024}.
\\



