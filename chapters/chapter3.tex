\chapter{Patentierbarkeit von KI generierten Erfindungen und Computerprogrammen \label{cha:chapter3}}


\section{Einführung in das Patentrecht\label{sec:Patentrecht}}


Das \gls{DPMA} ist die zentrale Behörde in Deutschland, 
die für den Schutz von geistigem Eigentum zuständig ist und 
beschreibt den Nutzen von Patenten wie folgt:
\\
\\
\textit{''Mit Patenten können Sie Ihre technischen Erfindungen 
(innovative Produkte oder Verfahren) vor unerwünschter Nachahmung schützen. 
Patente belohnen ihren Inhaber 
oder ihre Inhaberin durch ein befristetes und räumlich begrenztes Nutzungsmonopol.''} 
\footcite{DPMAPatentschutz}
\\
\\
Das \gls{BPatG} ist für Streitigkeiten über Patente zuständig, 
insbesondere für Nichtigkeitsklagen und Beschwerden gegen Entscheidungen des DPMA.

Das deutsche PatG
regelt die rechtlichen Rahmenbedingungen für Patente in Deutschland.
Das Patentrecht in Deutschland ist ein spezieller Teil des gewerblichen Rechtsschutzes, 
der wiederum zum Bereich des Immaterialgüterrechts gehört. 
Es ist durch das Grundgesetz geschützt, insbesondere durch Art. 14 GG, 
der das Eigentum und das Erbrecht gewährleistet. 
Im materiellen Sinne gehören Patente zum Eigentum.
Ein weiterer Teil des Immaterialgüterrechts ist das Urheberrecht. 
Dieses grenzt sich von anderen Teilen des
gewerblichen Rechtsschutzes wie das Markenrecht und Designrecht dadurch ab, das 
es kreative Leistungen schützt \footcite{GewerblicherRechtschutzUnd}. 


Ein Patent kann gemäß § 49 Abs.1 PatG von der Prüfstelle des DPMA erteilt werden. 
Um die Patentierbarkeit von durch KI geschaffener Computerprogramme 
zu prüfen fokussiert sich diese Arbeit auf die Erteilung von Patenten. 
Die Erteilung von Patenten lässt sich in formelle und materielle Vorrraussetzungen 
gliedern, was sich aus § 49 Abs.1 PatG ableitet. 
Das Deutsche Patent- und Markenamt prüft auf Antrag (§ 44 Abs.1 PatG) 
die formellen und materiellen Vorrausetzungen.

\section{Formelle Vorraussetzungen}

Zu den formellen Vorraussetzungen gehören:
\begin{enumerate}
    \item Anmeldung und Form, §§ 34, 37 und 38 PatG 
    \vspace{-0.11in} 
    \item Beseitigung gerügter Mängel, § 45 Abs. 1 PatG
\end{enumerate}

Patente müssen angemeldet werden (§ 34 Abs. 1 PatG). 
Gemäß § 34 Abs. 3 PatG muss die Anmeldung den Namen der/des Anmelders*in 
(Nr. 1), einen Antrag auf Erteilung des Patents, 
in dem die Erfindung kurz und genau bezeichnet ist (Nr. 2), 
einen oder mehrere Patentansprüche (Nr. 3), 
eine Beschreibung der Erfindung (Nr. 4) sowie die Zeichnungen, 
auf die sich die Patentansprüche oder die Beschreibung beziehen (Nr. 5), 
enthalten. 
Außerdem muss die Erfindung vollständig und deutlich offenbart sein (§ 34 Abs. 4 PatG) 
und nur eine einzige Erfindung enthalten (§ 34 Abs. 5 PatG). 
Paragraph 37 des PatG befasst sich mit der korrekten Erfinderbennenung 
und Paragraph 38 mit Änderungen der Anmeldung. 
\\

Sind die oben genannten formellen Anforderungen nach § 45 Abs. 1 PatG, 
nicht erfüllt wird der Anmelder aufgefordert, 
diese innerhalb einer bestimmten Frist zu beseitigen.

Wenn die gerügten Mängel beseitigt wurden 
oder es gemäß §§ 34, 37 und 38 PatG keine gerügten Mängel gibt 
sind die formellen Vorrausetzungen für die Erteilung eines Patents erfüllt.

Bei der Patentierbarkeit von KI generierten Erfindungen ist jedoch 
ein besonderes Augenmerk
auf § 37 PatG zu werfen, da hier keine natürliche Person
die Erfindung hergestellt hat sondern eine KI. 
Der Paragraph 37 Abs. 1 fordert eine Erfinderbenennung
innerhalb von fünfzehn Monaten.

\subsection{KI als Erfinder}
Bei Erfindungen, wo eine KI wesentlich
zur Entstehung beigetragen hat, stellt sich die Frage,
ob die KI als Erfinder benannt werden kann oder
ob der Nutzer der KI als Erfinder benannt werden muss.
Die Benennung der KI als Erfinder wird in Deutschland 
durch das DPMA geprüft. Bei der Entscheidung
ob eine KI als Erfinder benannt werden kann,
wird dabei auch auf das Vorgehen anderer 
Patentämter geschaut wie das EPA. 
Das PatG und das EPÜ weisen viele Gemeinsamkeiten
im Punkt der Erfinderbenennung auf, so sind 
§ 37 Abs. 1 PatG und 
Art. 81 Abs. 1 EPÜ fast identisch formuliert und 
fordern eine Benennug des Erfinders.
Ebenfalls entspricht § 7 PatV dem 
Art. 19 Abs.1 EPÜ in dem Angaben des Erfinders
bezüglich Familienname, Vorname und eine Adresse gefordert 
werden. 

\subsubsection{Europäische Rechtsprechung}
Am 17.10.2018 geht beim \gls{EPA} ein Patent ein, 
dass sich auf ein Design für einen speziellen Lebensmittelbehälter
bezieht,
welches von der KI "DABUS" (Device for the Autonomous Bootstrapping of Unified Sentience) 
entwickelt wurde. DABUS ist eine
KI, die von dem Forscher Dr. Stephen Thaler erstellt wurde
um Erfindungen zu generieren.
Bei der Anmeldung des Patents ist für Dr.Thaler 
keine natürliche Person als Erfinder
erkennbar und er lässt das Erfinderfeld
leer. Da die Erfinderbenennung gemäß Art. 81 Abs. 1 EPÜ verpflichtend
ist wurde die KI 
"DABUS"
nach Aufforderung vom EPA
als Erfinder ergänzt.
Das EPA hat daraufhin entschieden, 
dass diese Benennung nicht dem Artikel 81 Abs.1 \gls{EPÜ}  genügt,
da der Erfinder eine natürliche Person sein muss, sowie ein 
Familienname, Vorname und eine Adresse angegeben werden muss
gemäß Art. 19 Abs.1 EPÜ.
Damit sind die formellen Vorrausetzungen an das Patent mit dem Aktenzeichen
EP 18 275 163 (Food Continer) nicht erfüllt und das Patent wurde nicht erteilt. 
Außerdem wurde am 21.12.2021 entschieden, das Namen die Dingen gegeben werden nicht
mit Namen natürlicher Personen gleichzusetzen sind. Maschinen oder KI-Systeme 
haben keine Rechte, die durch den Namen ausgeübt werden können.
Das Recht auf ein europäisches Patent
ist in Art. 60 EPÜ geregelt \footcite{000820Designation}. 
\\
Am 20.Oktober 2020 wird das Thema KI-Patente im Europäischem
Parlament nochmals aufgegriffen und drei Berichte angenommen 
\footcite{KIRegelnWofuerEuropaeische2020}.
Dabei wird erkannt, 
dass die aktuelle Gesetzeslage nur Erfindungen berücksichtigt,
die von Menschen mit Hilfe von KI geschaffenen wurden. 
Eine klare Unterscheidung muss getroffen werden um diese 
von vollständig autonom von KI geschaffenen Erfindungen abzugrenzen
\footcite{TextsAdoptedIntellectual}.
Eine weitere Studie, welche von der EU Kommission beauftragt wurde 
sieht die derzeitige Gesetzeslage daraufhin als ausreichend 
und fordert erst Handlungsbedarf bei dem Einsatz 
von autonom handelnden KIs (starker KI).
Es wird vorgeschlagen dann ein spezifisches Roboterrecht zu schaffen, 
um den Umgang mit intelligenten Maschinen zu regeln 
und die Informationssicherheit zu gewährleisten
\footcite{gutaAPPLICABILITYGDPRARTIFICIAL2022}.
Jedoch ist es auch heute schon möglich teilweise 
Unabhängigkeit von menschlicher Einwirkung
zu schaffen, da auch schwache KIs durch Deep Learning 
Fähigkeiten nach Einwirken durch den Menschen erlangen, wie bei DABUS
\footcite{surdenMachineLearningLaw} \footcite{dornisDornisSchopfungOhne2021}.
Eine weitere Form autonomer KIs sind die Genetic Breeding Algorithmen,
welche
nur einen minimalen menschlichen Input beim Erschaffungsprozess 
verlangen, theoretisch danach jedoch ohne weiteren Input Erfindungen 
hervorbringen könnten.



\subsubsection{Deutsche Rechtsprechung}
Die Benennung der KI als Erfinder 
bei der Anmeldung einer Erfindung vor dem DPMA
ist ziemlich naheliegend, 
da in § 124 PatG zur Vollständigigkeit und Wahrheit vor dem
DPMA, 
dem Patentgericht und dem BGH aufgerufen wird.
Bei der Anmeldung vor dem DPMA am 17. Oktober 2019 mit dem Aktenzeichen
10 2019 128 120.2 kann die KI 
DABUS ebenfalls nicht als Erfinder in Kraft treten. 
Am 11.Juni 2024 hat der Bundesgerichtshof (BGH) ein Urteil gefällt 
( AZ X ZB 5/22).
Eine KI, 
erfüllt nicht die gesetzlichen Anforderungen an die Erfinderbenennung 
gemäß § 37 PatG. 
Nur eine natürliche Person darf 
als Erfinder benannt werden, 
dabei verweist das DPMA auf den überwiegenden Teil
der Fachliteratur und auf § 3 PatG.
In § 3 PatG steht, dass diejenige (natürliche) Person Als
Erfinder 
verstanden wird, 
deren schöpferischer Tätigkeit die Erfindung entspringt.
Zudem sind Pflichtangaben nach § 7 PatV, wie 
Familienname, Vorname und eine Adresse, 
des Patentanmelders gefordert,
welche eine KI nicht besitzt. 
Ein weiterer vom DPMA aufgeführter Grund für eine 
Bennenung einer natürlichen Person ist, dass 
die Stellung des Erfinders zusätzlich zur Erfindertätigkeit
auch rechtliche Beziehungen beinhaltet gemäß § 6 PatG.
Paragraph sechs des Patentgesetzes regelt das Recht auf
das Patent für seinen Erfinder.
Die KI "DABUS" wird vom DPMA als 
Sache bzw. Machine angesehen, 
welche kein Träger von Rechten sein kann.
Ein erster Hilfsantrag, dass keine Erfinderbenennung notwendig 
ist wurde abgelehnt aus diesen Gesichtspunkten.
Ein zweiter Hilfsantrag, in dem sich der KI Nutzer
Dr. Stephen Thaler als Erfinder einträgt, 
jedoch in der Bescheibung ergänzt, dass die Erfindung
einer KI entspringt wird ebenfalls abgelehnt.
Die Begründung hierfür ist, dass dieser Satz in der 
Beschreibung die korrekte Erfinderbenennung nach
§ 37 Abs. 1 PatG in Frage stellt.
Im dritten Hilfsantrag tritt Dr. Stephen Thaler
als Erfinder in Erscheinung, mit dem 
Zusatz, dass er die KI DABUS dazu beauftragt hat,
die Erfindung zu schaffen.
Dieser Antrag wurde vom DPMA angenommen,
da die KI hier als Hilfsmittel angesehen wird,
welches durch den Menschen bedient wird.
Wenn der Anmelder der einzige ist, 
der die Maschine genutzt hat, 
und keine andere Person zur Erfindung beigetragen hat, 
kann er sich selbst als Erfinder benennen. 
Die zusätzliche Angabe, dass eine KI, die Erfindung
generiert hat stellt nicht die rechtmäßige
Angabe des Erfinders in Frage, da die KI hier nur als 
Werkzeug gesehen wird und das Verwenden ähnlich dem 
Einsatz traditioneller Hilfsmittel zu werten ist.
Falls der Anmelder Bedenken hat, 
die Nutzung der künstlichen Intelligenz zu verschweigen, 
kann er die Nutzung in der Beschreibung der Patentanmeldung angeben 
und so der Wahrheitspflicht gemäß § 124 PatG gerecht werden.
Ausgehend von diesen Grundsätzen genügt 
für die Stellung als Erfinder bei einer technischen Lehre, 
die mit Hilfe eines Systems der künstlichen
Intelligenz aufgefunden wurde, 
ein menschlicher Beitrag, 
der den wesentlich Gesamterfolg beeinflusst hat. 
Da derzeit jedoch noch kein KI System eine Erfindung ohne
Erfinder (Nutzer) generieren kann ist jeder Beitrag Als
wesentlich zu betrachten \footcite{zivilsenatZB222024}.
\\

\subsubsection{Exkurs: Internationales Patentrecht\label{sec:intp}}
Die WIPO(World Intellectual Property Organization) 
hat die KI DABUS als Erfinder zugelassen,
da sie keine spezifischen gesetzlichen Vorschriften hat
, welche dagegensprechen könnten. 
Das WIPO
selbst vergibt jedoch keine Patente, 
sondern koordiniert das PCT-Verfahren, 
das es Erfindern ermöglicht, 
in mehreren Ländern gleichzeitig 
Patentanmeldungen zu stellen. 
Somit ist es
von den nationalen und regionalen Patentämtern abhängig
ob eine KI als Erfinder benannt werden kann.
Auf internationaler Ebene werden ebenfalls Erfinderbenennung verlangt,
welche von einer natürlichen Person ausgehen. So hat 
der UK Supreme Court am 20. Dezember 2023 das Patent von mit 
Erfinderbenennung "DABUS" abgelehnt, der Court of Appeal for England
and Wales am 21. September 2021,
der Federal Court of Australia am 13.April 2022,
der United States Court of Appeals for the
Federal Circuit am 5. August 2022 
und  der High Court of New
Zealand am 17. März 2023.
Die Entscheidungen sind dabei ähnlich wie die 
in Deutschland und Europa.
Einzig das Patentamt aus 
Südafrika hat die KI DABUS als Erfinder anerkannt.
Südafrika hat keine gesetzlichen Vorschriften, 
die festlegen, 
dass ein Erfinder ein Mensch sein muss und im Jahr 2021 das
erste Patent an eine KI erteilt.


\subsubsection{De lege lata}
Nach geltendem Recht "de lege lata" ist es in Deutschland und Europa
derzeit nicht möglich eine KI 
als Erfinder in Erscheinung treten zu lassen.
Die Gesetzeslage sieht vor, 
dass ein Erfinder eine natürliche Person sein muss,
dabei ist es egal, ob eine "schwache KI" oder eine 
bisher noch fiktive "starke KI" verwendet wird.
Als Erfinder tritt der Nutzer der KI ein, 
welcher diese dann offiziell
als Werkzeug benutzt,
um eine technische Erfindung zu produzieren.


\subsubsection{De lege ferenda}
Nach zukünfigem Recht "de lege ferenda" könnte sich jedoch einiges tun.
Die Debatte wird von verschiedenen 
Instanzen wie dem europäischen Parlament immer wieder aufgenommen und es 
werden in Zukunft weitere Fälle von Erfindungen folgen die eine klare rechtliche 
Klärung bedürfen. 
Bei den bisherigen schwachen KIs ist es noch möglich die KI als Werkzeug
zu sehen, wobei es dort schon Schwierigkeiten geben könnte, wenn eine KI,
welche durch einen Genetic Breeding Algorithmus erstellt wurde eine Erfindung
schafft, ohne vorher einen Input bekommen zu haben (Training ausgenommen).
Ab dem Punkt, wo kein Mensch mehr im Erfindungsprozess beteiligt,
sondern nur beim Entwicklungsprozess der KI beteiligt ist,
braucht es auf jeden Fall ein zusätzliches Recht.
Einen Ansatz bieten Konertz und Schönhof mit einem 
„erfinderloses Patent“, das es der Person, 
die das Computersystem nutzt, erlaubt, 
die Rechte an der durch die Maschine generierten Erfindung zu beanspruchen
\footcite{konertzErfindungenDurchComputer2018}. 
Dieser Ansatz ist eine Fortführung der derzeitigen Situation mit ein paar
juristischen Anpassungen, welcher jedoch in den nächsten Jahrzehnten 
an seine Grenzen kommen wird.
Spätestens die ersten starken KIs werden eine klare rechtliche Abgrenzung 
brauchen, 
wenn Erfindungen komplett autonom 
von einer KI erschaffen werden und ein "erfinderloses Patent" 
nicht mehr ausreicht.
Dann stellt sich die Frage ob die KI doch als Erfinder auftreten kann 
und eigene Rechte besitzen darf. Ähnlich wie in dem Entwurf des Europäischen
Parlaments vom 31.5.2016 mit
Empfehlungen an die Kommission zu zivilrechtlichen Regelungen im
Bereich Robotik \footcite{delvauxMitEmpfehlungenKommission}. 
Dieser schlägt 
eine Einführung eines eigenen Rechtsstatus 
für Roboter als "elektronische Personen" vor.
Roboter können danach Rechte und Pflichten haben, 
ähnlich wie Menschen.
Eine andere Möglichkeit wäre die Einführung eines wie oben 
erwähnten Roboterrechts, welches die Rechte und Pflichten von
autonomen Maschinen regelt und so auch die Erfinderbenennung
von KIs regelt.
\\

\section{Materielle Vorrausetzungen}

Die materiellen Vorraussetungen der Patentanmeldung sind 
in den Paragraphen §§ 1 – 5 PatG geregelt 
und lassen sich in folgende Punkte unterteilen :

\begin{enumerate}
    \item Erfindung auf einem Gebiet der Technik, § 1 Abs. 1 PatG
    \begin{enumerate}
    \vspace{-0.05in}
    \item Ausschluss, §§ Art. 1 Abs. 3 PatG und § 1 Abs. 4 PatG
    \end{enumerate}
    \vspace{-0.11in} 
    \item Neuheit, § 1 Abs. 1 i.V.m. § 3 PatG
    \vspace{-0.11in} 
    \item Erfinderische Tätigkeit, § 1 Abs. 1 i.V.m. § 4 PatG
    \vspace{-0.11in} 
    \item Gewerbliche Anwendbarkeit, § 1 Abs. 1 i.V.m. § 5 PatG
    \vspace{-0.11in} 
    \item Ausschluss, § 2 PatG
\end{enumerate}

\subsection{Technische Erfindung \label{sec:tecer}}

Im ersten Absatz des ersten Paragraphen im Patentgesetz wird festgelegt,
dass eine Erfindung auf einem Gebiet der Technik liegen muss, neu sein und
gewerblich anwendbar.
Eine technische Erfindung liegt laut Haedicke dann vor, 
wenn die Erfindung aus dem Bereich der Physik, 
Chemie oder den Ingenieurswissenschaften ist, 
welche sog. Gebiete der Technik darstellen.
In den letzen Jahrzehnten hat sich der Begriff 
"Technik" auch auf Erfindundungen der Biotechnologie, 
Telekommunikations- und Computertechnologie ausgeweitet 
\footcite{haedickeEinfuhrung2020}
Der \gls{BGH} erstellte die
sog. „Rote-Taube“-Formel, welche technisch als  
„eine Lehre zum planmäßigen Handeln 
unter Einsatz beherrschbarer Naturkräfte zur Erreichung eines 
kausal übersehbaren Erfolgs definiert.“\footcite{BGH27031969}  
Dabei genügt es „wenn die beanspruchte Lehre den Einsatz technischer Geräte umfasst“
\footcite{BGH3020152015}\footcite{BGH2420112011}. 
Aufgrund der ständigen Entwicklung lässt sich 
jedoch der Begriff der „technischen Erfindung“ 
nicht abschließend definieren \footcite{haedickeEinfuhrung2020}.
\\
In Paragraph 1 Abs. 3 PatG werden Gegenstände und Tätigkeiten festgelegt, 
die nicht als Erfindung angesehen werden dürfen. 
Diese sind nicht patentfähig als solche (§ 1 Abs. 4 PatG). 
Ausgeschlossene Erfindungen sind Entdeckungen, 
sowie wissenschaftliche Theorien und mathematische Methoden, 
ästhetische Formschöpfungen, Pläne, Regeln und Verfahren für gedankliche Tätigkeiten, 
für Spiele 
oder für geschäftliche Tätigkeiten sowie Programme für Datenverarbeitungsanlagen, 
sowie die Wiedergabe von Informationen (§ 1 Abs. 3 PatG). 
\\

\subsection{Patentierbarkeit von Computerprogrammen}
Das Programme für Datenverarbeitungsanlagen, sowie mathematische Methoden grundsätzlich
von der Patentierbarkeit ausgeschlossen sind, stellt Schwierigkeiten in der 
Patentierbarkeit von durch künstliche
Intelligenz geschaffener Computerprogramme dar. 
Jedoch betrifft der Ausschluss nur Programme "als solche", 
was bedeutet, 
dass Computerprogramme in bestimmten Zusammenhängen patentierbar sind.
Dies ist der Fall,
wenn ein Computerprogramm eine technische Aufgabe löst
und damit technische Merkmale aufweist.
Beispiele für patentierbare Software sind technische Anwendungsprogramme, 
die Messergebnisse verarbeiten, 
technische Einrichtungen überwachen oder in technische Systeme eingreifen
\footcite{RedekerITRechtSchutz}.
Bevor Beispiele von patentierbaren Computerprogrammen folgen, ist es
nötig den Begriff des Computerprogrammes eindeutig zu definieren.
\subsubsection{Definition: Computerprogramm}
Ein Computerprogrammm ist laut ISO/IEC 2382-1:1993 
eine Kombination von Anweisungen und Deklarationen in einer
Programmiersprache, die einen Computer dazu bringen, 
Funktionen zur Lösung eines Problems auszuführen 
\footcite{instituteofelectricalandelectronicsengineersinc.ISO47652010}.
Das in der Programmiersprache geschriebene Programm(Quellprogramm) wird 
mittels eines Sprachcompilers in ein in Maschinensprache geschriebenes 
Programm(Objektprogramm) aus Nullen und Einsen umgewandelt \footcite{WasIstProgramm}.
Computerprogramme werden nach dem Wirtschaftslexikon Gabler in Systemprogramme
und Anwendungsprogramme unterteilt. 
Ein Anwendungsprogramm löst dabei eine bestimmte Aufgabe des Anwenders, 
wie z.B. Ticketreservierungen oder Parkraumüberwachung 
\footcite{lackesDefinitionAnwendungsprogramm}. 
Während ein Systemprogramm ein Bestandteil des Betriebssystems ist und 
für den Nutzer nicht sichtbare Teile der internen Steuerung des Computer übernimmt,
wie die Orchestrierung der als nächstes 
zu bearbeitenden Aufgabe \footcite{lackesDefinitionSystemprogramm}.
Für beide Arten von Programmen gibt es mögliche Ausprägungen, 
welche patentierbar sein können, 
so kann ein mögliches Szenario bei einem Anwendungsprogramm sein,
dass ein Algorithmus entwickelt wird der in der Branche noch nicht vorhanden ist. 
Z.B. ein
Bildverarbeitungsprogramm, das einen neuen Algorithmus verwendet, 
um Bilder zu verbessern.
Oder im Falle von Systemprogrammen ein Betriebssystem mit einem 
innovativen Verfahren zur Erkennung und Verhinderung von Malware.

\subsubsection{Abgrenzung zur Software}
Ein Begriff der heutzutage oft synonym zu dem Begriff Computerprogramm benutzt wird,
ist Software. Software ist 
laut ISO/IEC 2382-1:1993 eine Sammlung von Computerprogrammen, 
Daten und Bibliotheken und 
somit ist ein Computerprogramm nur ein Bestandteil einer Software.

Software verwendet Computerprogramme 
als Tools um individuelle Anweisungen auszuführen 
\footcite{ComputerProgrammeUnverzichtbareComputerprogramme}.
Software wird nach dem oben genannten ISO-Standard außerdem in 
Systemsoftware, Unterstützungssoftware und Anwendungssoftware unterteilt.
Systemsoftware ist Software, welche die Hardware des Computers steuert, 
wie z.B. Betriebssysteme oder Gerätetreiber. Unter Unterstützungssoftware 
fällt Software, die die Entwicklung und Ausführung von Anwendungssoftware
unterstützt, wie z.B. Compiler oder Texteditoren. Anwendungssoftware ist
Software, die für die Lösung von Problemen oder die Durchführung von Aufgaben
entwickelt wird, wie z.B. Textverarbeitungsprogramme oder Spiele
\footcite{instituteofelectricalandelectronicsengineersinc.ISO47652010}.
\\

\subsubsection{Urteile mit Bezug auf Patentierbarkeit von Computerprogrammen}
Es gibt viele Präzendenzfälle in den Computerprogramme patentiert worden sind, 
sowohl in Deutschland vor dem DPMA als auch in Europa vor dem EPA.
Der in § 1 Abs. 3 des deutschen PatG Ausschluss für Computerprogramme als 
solche(Programme für Datenverarbeitungsanlagen) ist im EPÜ in Art. 52 (2) 
und Art. 52 (2) fast im selben Wortlaut geregelt.

\paragraph{Europäische Rechtsprechung}
Das Europäische Patentamt hat mit dem Patent EP0005954 
„Verfahren und Vorrichtung zur verbesserten digitalen Bildverarbeitung“,
welches am 22.05.1979 angemeldet wurde,
einen entscheidenden Meilenstein gesetzt, 
der den Weg für die Patentierbarkeit mathematischer Methoden und 
Computerprogramme geebnet hat. 
Am 15.6.1986 wurde mit der Entscheidung T0208/84 das
erste Mal in Europa ein Patent für ein Computerprogramm erteilt.
Laut Regel 29 (1) sind bei Erfindungsanmeldungen vor dem EPA
immer die 
technischen Merkmale der Erfindung anzugeben.
Das hier aufgeführte Computerprogramm ist ein 
Verfahren zur digitalen Bildverarbeitung, dass
auf einem mathematischen Algorithmus basiert \footcite{EPThisFile}.
Der technische Charakter liegt in der Verbesserung 
der Bildqualität durch ein spezielles Filterverfahren.
Bei dem sog. Vicom-Fall,
hat das EPA eine Patentanmeldung einer
mathematische Methoden genehmigt, da sie in einem technischen Kontext 
verwendet wird. 
Es wurden grundlegende Leitsätze für die Patentierbarkeit von
Computerprogrammen festgelegt und entschieden, dass mathematische
Methoden als Bezugsrahmen für technische Erfindungen zulässig sind.
Der Schutz gilt hier nicht dem Computerprogramm,
sondern der Anwendung eines Programmes 
zur Festlegung der Schrittfolge in dem Verfahren.
Dieser Fall war ausschlaggebend für das Patentvorgehen
in den 1980er bis 1990er Jahren und definierte grundlegend,
ab wann ein Computerprogramm als technisch angesehen wird.
Durch mehrere Entscheidungen rund um 2000, vorallem aber den 
Comvik Fall hat sich die Bewertung der Patentierbarkeit von
Computerprogrammen weiter verfeinert \footcite{ComvikAnsatz}.
Der Comvik Fall \footcite{064100Zwei} betrifft ein Patent 
über ein digitles Mobilfunksystem, welches sowohl technische,
als auch nicht technische Merkmale aufweist. 
Dabei hat das EPA am 26.09.2002 in der
Entscheidung T0641/00 festgelegt, dass für die Feststellung der
erfinderischen Tätigkeit gemäß Art.56 EPÜ bei Patenten nur die 
Merkmale mit technischen Charakter in das Patent einfließen.
Das bedeutet, dass in der Formulierung eines Schutzanspruches
nicht technische Bestandteile enthalten sein dürfen, 
diese sich jedoch aus der Bewertung des Patents gestrichen werden.
So kann ein Schutzanspruch, welcher eine Mischung aus technischen 
Merkmalen und nicht technische 
Merkmalen enthält, aber nur nicht technische Neuerungen aufweist 
als nicht patentierbar angesehen werden aufgrund von fehlender
erfinderischer Tätigkeit. Nicht technische Merkmale,
die zum technischen Charakter Beitragen werden jedoch bewertet.
Das ist insofern relevant für Computerprogramme,
dass nur solche Computerprogramme als patentierbar anerkannt werden, 
die über reine Geschäfts- oder Verwaltungsideen hinausgehen.
Der Comvik-Ansatz wird vom EPA seit Anfang der 2000er Jahre 
konsequent angewandt und 
wurde durch viele Folgefälle bestätigt.
Der Hauptunterschied zum Vicom Ansatz liegt in der
Bewertung der erfinderischen Tätigkeit, welche durch
die früher getrennte Beurteilung mit 
der Patentierbarkeit aufgrund eines "technischen Beitrags"
einen gewissen Spielraum für Verwirrungen bot.
Das EPA und das DPMA haben ähnliche Rechtsgrundlagen
jedoch lässt das DPMA in bestimmten Fällen 
eine flexiblere Auslegung des Gesetzestextes zu, 
insbesondere wenn es um Computerprogramme geht.
Das Europäische Patentamt (EPA) hat eine strengere Auslegung der Patentierbarkeit 
bei Erfindungen von Computerprogrammen. Es folgt dem Comvik-Ansatz
sodass,
wenn eine Hardware-Optimierung eine 
Kombination aus Hardware- und Softwareverbesserungen beinhaltet, das 
EPA strenger bewerten könnte, da es nur die technischen Beiträge berücksichtigt.
Die deutsche Rechtssprechung verfolgt hingegen nicht den Comvik-Ansatz.

\paragraph{Deutsche Rechtsprechung\label{par:deur}}
Die erste deutsche Rechtssprechung für computerimplementierte Erfindungen
geht auf die 
BGH Entscheidung X ZB 23/74 vom 22.06.1976 zum Dispositionsprogramm zurück 
\footcite{BGH22061976}. 
Dabei ist der erste wichtige Leitsatz, dass es nicht ausreicht technische Mittel
zu verwenden damit eine Erfindung als technisch angesehen wird
, sondern das Verwenden technischer Mittel muss ein Bestandteil der Problemlösung sein.
Derzufolge kann eine Kombination aus nicht technischen und technischen 
Merkmalen technisch sein kann, wenn ihr sachlicher Gehalt technisch ist.
Außerdem wird festgelegt, dass ein Computerprogramm patentierbar sein
kann, wenn es den Computer auf eine neue nicht naheliegende Weise nutzt.
Dies bildet die erste Vorraussetung zur Patentierbarkeit von Computerprogrammen
in Deutschland. 
Der zweite wichtige Leitsatz zur Technizität von Computerprogrammen ist,
dass die neuen und erfinderischen Teile 
einer Erfindung auf dem Gebiet der Technik liegen müssen.
Das erste deutsche Patent für eine computerimplementierte Erfindung bildet 
das Antiblockiersystem\footcite{Bundesgerichtshof13051980}.
Dabei beinhaltet dieses Patent zwar ein Computerprogramm,
als wesentliche Komponente, trotzdem wurde hier die technische Funktion,
die von dem Computerprogramm unterstützt wurde patentiert.
Die Anwendung des zweiten Leitsatzes aus dem Urteil zum Dispositionsprogramm, 
nach der nur neue und erfinderische Aspekte auf Technizität geprüft werden sollen,
wurde bei diesem Urteil von dem BGH aufgelöst. 
Die Kombination aus den beiden nicht patentierbaren Bestandteilen
des ABS, die Bremsen und das Computerprogramm ergeben zusammen 
das patentierbare ABS. 
Das automatische, elektronische Steuern des Bremsvorgangs gab es vorher 
in dieser Weise noch nicht.
Der erste Leitsatz aus dem Urteil Dispositionsprogramm wurde
erstmals 1992 durch die BGH-Entscheidung "Tauchcomputer"
aufgegeben \footcite{BGH04021992}. 
Der BGH definiert in der Entscheidung, das der gesamte Anspruchsgegenstand
eines Patentantrags geprüft werden soll, 
einschließlich nicht technischer Merkmale,
wenn diese mit technischen verknüpft sind.
Die Entscheidung Logikverifikation 1999 \footcite{BGH13121999}
hat diesen endgültig abgeschafft und
sich auf die in der "Rote Taube" Formel
berufen, nach der die Definition für Technizität 
an die Entwicklung von Technik angepasst werden müsse.
Nun konnten auch rein computerimplementierte Erfindungen 
rechtlich als Technologie eingeordnet werden, solange 
zumindest teilweise ein 
“Einsatz beherrschbarer Naturkräfte zur Erreichung eines kausal übersehbaren Erfolges”
vorliegt. Seitdem gilt,
das computerimplementierte Erfindungen patentierbar sind
solange sie "ein konkretes Problem mit technischen Mitteln" lösen.
Das DPMA hat zur Prüfung des technischen Charakters eines 
Computerprogrammes einen dreistufigen Prüfungsansatz entwickelt \footcite{DPMAPatentschutz}.
Der erste Schritt ist die Bestimmung des technischen Beitrags,
also ob eine Erfindung auf einem technischen Gebiet liegt. 
Dies ist auch schon implizit bei jedweiliger Nutzung einer Datenverarbeitungsanlage
gegeben, da mit einer technischen Maschine zusammengewirkt wird
und somit bei fast allen computerimplementierten Erfindungen gegeben.
Der zweite Schritt ist die Prüfung, ob die gesamte Erfindung ein technisches Problem löst.
Das deutsche und das europäische Patentamt sind sich einig,
dass die bloße Implementierung 
einer Rechenmethode auf einem Computer, ihr
ihr noch keinen technischen Charakter verleiht. 
Entscheidend ist die konkrete Anwendung der Methode 
in einem technischen Zusammenhang, der einen unmittelbaren Effekt in der 
physischen Welt erzeugt \footcite{melullisEPUArt522023}.
So sind mehrere Patente abgelehnt worden,
wie, Verfahrenen, 
die lediglich der Auswertung von Daten auf statistischer Basis dienen.
Dies war der Fall bei den Beschlüssen 17 W (pat) 74/07\footcite{BPatG10012012}
und 17 W (pat) 6/00 \footcite{BPatG01032001} vom Bundespatentgericht.
Das EPA sieht analog Verfahren zum Sammeln und Auswerten von Daten 
im Rahmen von betriebswirtschaftlichen Prozessen als nicht patentfähig an 
(siehe T 154/04 \footcite{EuropaischesPatentamt152006}),
da sie kein technisches Problem lösen.
Im dritten Schritt wird erfinderische Tätigkeit und Neuheit 
geprüft, wobei nur Aspekte berücksichtigt werden,
die die Lösung des technischen Problems beeinflussen.
Dies ist ähnlich dem Comvik Ansatz unterscheidet sich
aber dahingehend, dass hier auch ein nicht technisches
Merkmal einen Einfluss auf die Patentierbarkeit haben kann,
wie in der Entscheidung "Tauchcomputer" festgelegt.
\\
Nach der Definition von Computerprogrammen und der Klärung von dem Begriff
"Technik" kann nun eine Abschätzung getroffen werden, 
ab wann Computerprogramme patentierbar sind nach den
unterschiedliche Auslegung der Gesetzestexte vom EPA und DPMA. 
Durch die verschiedenen Urteile wird außerdem sichtbar 
ab wann ein Computerprogramm als technisch angesehen wird.
Jedoch bleibt die Patentierbarkeit von Computerprogrammen 
ein komplexes und umstrittenes Thema 
und internationale Standards könnten 
langfristig mehr Klarheit und Konsistenz schaffen.
Viele Computerprogrammpatente wurden abgelehnt und erst nach 
einer Beschwerde und erneuter Prüfung erteilt.
Wege um das Patentgesetz dahingehend zu vereinfachen schlägt 
Prof. Dr. jur. Dr. rer. pol. Jürgen Ensthaler 
vor. 
Eine mögliche Lösung wäre, die „als solche“-Formel durch eine Regelung zu ersetzen, 
wie sie für Gensequenzen in § 1a PatG besteht. 
Demnach könnten Algorithmen nur dann patentiert werden, 
wenn die konkrete technische Funktion klar benannt 
und in den Patentanspruch aufgenommen wird. 
Diese Funktionsbegrenzung würde verhindern, 
dass abstrakte mathematische Lehren, 
die für viele Anwendungen nutzbar sind, patentiert werden 
\footcite{ensthalerEnsthalerBegrenzungPatentierung2013}. 
\subsection{Exkurs: Urheberrecht\label{sec:urh}}
Während das Patentrecht technische Lösungen, 
insbesondere innovative Verfahren, 
Systeme oder Methoden, die auf einem Computer implementiert werden,
im Bereich von computerimplementierten Erfindungen
schützt, schützt das Urheberrecht die kreative Leistung des Urhebers.
Das Urheberrecht schützt die konkrete Ausdrucksform von Software, 
wie den Quellcode und den Objektcode. 
Es schützt die kreativen und individuellen Aspekte der Software, 
nicht jedoch die dahinterliegende Funktion oder Idee.
Zum Beispiel ist
der Quellcode eines Textverarbeitungsprogramms
urheberrechtlich geschützt, 
aber nicht die Idee, Textverarbeitung als solche.
So ist Microsoft Word urheberrechtlich geschützt, d.h.
der Quellcode ist geschützt,
der die Funktionen der Textverarbeitung steuert,
Die Idee, Text zu verarbeiten, ist jedoch nicht geschützt.
Das Urheberrecht entsteht automatisch mit der Schöpfung 
und hält für die Lebensdauer des Autors plus 70 Jahre.
Zusammengefasst schützt das Urheberrecht den kreativen Aspekt der Software (Code), 
während computerimplementierte Patente 
den technischen Beitrag oder die technische Innovation schützen, 
die durch die Software ermöglicht wird.

\subsection{Neuheit}
Eine Erfindung gilt als neu, wenn sie nicht zum Stand der Technik gehört. 
Der Stand der Technik umfasst alle Kenntnisse, 
die vor dem für den Zeitrang der Anmeldung
maßgeblichen Tag durch schriftliche 
oder mündliche Beschreibung, durch Benutzung oder in
sonstiger Weise der Öffentlichkeit zugänglich gemacht worden sind (§ 3 Abs. 1 PatG).
Als Stand der Technik gilt auch der Inhalt nationaler Patentanmeldungen 
in der beim Deutschen
Patentamt ursprünglich eingereichten Fassung mit älterem Zeitrang, 
die erst an oder nach dem für
den Zeitrang der jüngeren Anmeldung maßgeblichen Tag der Öffentlichkeit 
zugänglich gemacht
worden sind (§ 3 Abs. 2 Nr. 1 PatG). 
Bei KI-generierten Erfindungen kann es schwierig sein, 
den Stand der Technik umfassend zu bestimmen. 
KI-Modelle können auf umfangreiche Daten zugreifen und Lösungen generieren, 
die in kleinen Teilen bereits veröffentlicht, 
aber in dieser spezifischen Kombination noch nicht dokumentiert sind.
Im Kontext von KI stellt sich hier die Frage 
ob eine Erfindung überhaupt als neuartig angesehen werden kann, 
da KI basierte Erfindungen meist aus der Analyse großer Datenmengen bestehen
und Output auf Grundlage dieser Daten generieren. 
Es werden nur bestehende Muster im Stand der Technik kombiniert 
und so durch Kombination aus anderen konkreten technischen Lösungen, 
welche bereits in irgendeiner Form öffentlich zugänglich waren, ein neue generiert.
Eine Kombination bekannter technischer Lösungen kann als „neu“ angesehen werden,
wenn genau diese spezifische Kombination zuvor noch nicht offengelegt wurde,
siehe ABS aus Kapitel \ref{par:deur}.
Eine Neuordnung von Informationen stellt keine Neuheit dar.
Bei einer Kombination von 
bestehenden Erfindungen stellt sich jedoch die Frage 
ob die Kombination naheliegend ist und
somit keine erfinderische Höhe aufweist. 
\todo{Neuheit wenn Infos in ChatGPT}


\subsection{Erfinderische Tätigkeit}

%spannnend da zum ersten mal kombi aus KI und computerimplementiert

Eine Erfindung gilt als auf einer erfinderischen Tätigkeit beruhend, 
wenn sie sich für den Fachmann
nicht in naheliegender Weise aus dem Stand der Technik ergibt (§ 4 S. 1 PatG). 
Gehören zum Stand
der Technik auch Unterlagen im Sinne des § 3 Abs. 2 PatG, 
so werden diese bei der Beurteilung der
erfinderischen Tätigkeit nicht in Betracht gezogen (§ 4 S. 2 PatG).

\subsubsection{Verfahrenspatente}

Wenn der Einsatz einer bestimmten KI in der Branche üblich ist 
und diese benutzt wird,
ist nicht mehr von einer Erfindungshöhe zu sprechen, argumentiert
die Juristin und Juniorprofessorin Tochtermann von der Uni Mannheim.
So der Fall, wenn in einem bestimmten Bereich die Verwendung von
Machine Learning Standard ist und man patentieren lassen will,
dass man in diesem Bereich Machine Learning verwendet \footcite{WieManPatente2021}.
Die erfinderische Tätigkeit ist so leicht zu reproduzieren,
dass eine erfinderische Höhe als fraglich angesehen wird.
Somit stellt sich die Frage: 

Ist nun auch alles dem Stand der Technik zugehörig, was durch eine KI aus dem 
Stand der Technik hergeleitet werden kann? 

Dr. Joel Nägerl, Dr. Benedikt Neuburger und Dr. Frank Steinbach beschäftigen
sich mit dieser Thematik und kommen zu dem Schluss,
dass der Fachmann, der Beurteilungen vornimmt, 
in Zukunft KI-gestützte Systeme als „Lesebrille“ verwenden könnte.
Dies könnte dazu führen, dass Erfindungen, 
die heute noch mit einer erfinderischen Tätigkeit bewertet werden, 
in der Zukunft als naheliegend betrachtet und somit nicht patentierbar wären.
Dabei ist jedoch entscheidend, 
dass nur solche KI-Systeme herangezogen werden dürfen, 
die zum Zeitpunkt der Patentanmeldung verfügbar und gängig sind. 
Es ist unzulässig, den Bewertungsmaßstab durch die nachträgliche Nutzung 
einer später entwickelten, leistungsstärkeren KI zu erhöhen 
\footcite{nagerlKunstlicheIntelligenzParadigmenwechsel2019}.

Der Gesetzestext gibt ebenfalls Grund zum Anlass die erfinderische 
Tätigkeit von KI-generierten Erfindungen in diesem Kontext anzuzweifeln.
Der im Gesetzestext erwähnte Fachmann ist eine fiktive Person
und bildet im patentrechtlichen Sinne das Wissen der
Fachleute in dem jeweiligen Gebiet der Erfindung tätigen Industrie ab,
welche in der Entwicklung tätig sind \footcite{asendorfPatGErfindungAuf2023}.
Wenn KI-Systeme in einer bestimmten Branche üblich 
und als Werkzeug anerkannt sind, kann argumentiert werden, 
dass das durch die KI erlangte Wissen 
zum allgemeinen Wissen der Fachleute gehört. 
Das würde bedeuten, dass der Begriff des Fachmannes
in dieser Branche auch die Verwendung von KI einschließt, 
um Probleme zu lösen oder Entwicklungen zu erkennen.

Auch wenn in einer Branche KI zum Standardwerkzeug gehört
könnte dort die Anwendung von KI patentiert werden.
Der zentraler Aspekt 
des § 4 PatG ist die menschliche erfinderische Tätigkeit.
Ein Ansatz die Patentierbarkeit bei der Benutzung 
von KI zu bewerten in einer Branche in der die Anwendung 
üblich ist wäre, ähnlich wie im Comvik-Ansatz
der technische
und nichttechnische Merkmale getrennt sieht
und dabei nur die technischen bewertet, dies auch 
auf Erfindungen im Zusammenhang mit KI zu übertragen.
Durch die Trennung der menschlichen Leistung und 
der Leistung durch die KI kann die erfinderische Tätigkeit 
durchaus sinnvoll bewertet werden.
So wäre dann hier der Weg zu der Lösung patentierbar, anstatt der
Einsatz der KI.
Die eigentliche erfinderische Tätigkeit liegt dann im Umgang mit der KI 
und ihrer spezifischen einzigartigen Anwendung auf ein technisches Problem.
Der kreative Einsatz der KI wird dadurch 
als technische Maßnahme betrachtet. 
Dies würde dann in die Kategorie der Verfahrenspatente fallen,
ähnlich wie im Falle der Funktionsüberwachung für KI-Module
der Robert Bosch GmbH \footcite{DPMAregisterOriginaldokumentDE102017212328A1}
Wenn in einer besonders kreativen und für den Fachmann
nicht naheliegenden Weise mit der KI umgegangen wird, stellt 
dies eine erfinderische Tätigkeit dar.
\\
\subsubsection{Erzeugnispatente}
Die Frage, ob ein Produkt der KI patentierbar ist, ist
meist schwierig zu beantworten.
In vielen Fällen agiert die KI als eine Art „Black Box“, 
bei der die Inputs vom Menschen bereitgestellt werden, 
die endgültigen Outputs jedoch nicht vollständig nachvollziehbar sind 
\footcite{pauliniKIgenerierteErfindungPatentrechtliche}. 
Um die erfinderische Tätigkeit vollumfassend beurteilen zu können
muss eindeutig sein, wie die KI zu ihrem Ergebnis kam. 
Trotzdem ist es möglich anhand des Gesetzestextes Abschätzungen
zu treffen, ob der Output der KI patentierbar ist.
Wenn der Output für den Fachmann nicht aus dem bereits bekannten 
ableitbar ist, ist nach § 4 S. 1 PatG eine erfinderische Tätigkeit gegeben.
Das bedeutet, dass ein Produkt welches durch eine KI generiert
wurde und nicht naheliegend ist durchaus patentiert werden kann.
Der Leiter des Teams für maschinelles Lernen bei Autodesk,
Mike Haley beschreibt, dass in der Luft-und Raumfahrttechnik,
eine Software unterschiedliche Entwurfsmöglichkeiten für optimale
Designs vorschlägt. Diese werden dann von den Ingenieuren interpretiert.
Diese mit Generativen Design erstellten Produkte
wiederum stellen einen Einsatz der KI als Hilfsmittel dar, da
ein Mensch die Ergebnisse des von der KI-entwickelten
Designs fachspezifisch untersuchen muss und seine Expertise dazu
gibt
\footcite{WieManPatente2021}. Der Einsatz von KI
als Hilfsmittel beeinträchtigt nicht die erfinderische Tätigkeit.
So werden für solche Designs durchaus Patente gegeben, 
wenn sie den anderen patentrechtlichen Vorrausetzungen genügen.
Diese Patente sind dann unter dem Punkt Erzeugnispatente zu 
finden, welche den Schutz von Gegenständen ermöglichen.
Wenn jedoch nur ein einziges Produkt erstellt wird, 
welches nicht nochmal von einem Ingenieur speziell 
einer Bewertung unterzogen wird in einer Branche, 
in der der Einsatz von KI als Standard gilt, weist 
dies keine große erfinderische Höhe auf. Das wäre in 
diesem Fall, in der Luft und Raumfahrttechnik,
wenn ein Design von der KI erstellt wird unter 
Verwendung der üblichen branchenüblichen Software
und dieses Design ohne Bewertung der Vorteile und 
Vergleich mit anderen Designs zur Patentanmeldung gebracht wid.
\subsubsection{Zusammenfassung}
Insgesamt ist die Bewertung der erfinderischen Tätigkeit
von KI-generierten Erfindungen derzeit stark von der juristischen Auslegung abhängig. 
Solange der Beitrag der KI nicht klar und nachvollziehbar ist, 
bleibt es schwierig, 
die erfinderische Tätigkeit von durch KI produzierten Output
im patentrechtlichen Sinne zu bewerten.
Die Gesetzeslage könnte sich dementsprechend weiterentwickeln, 
die Reichweite 
vom Stand der Technik auf den Entwicklungshorizont von KI auszuweiten.
Dies ist bei dem derzeitigen Stand von KI durchaus sinnvoll.
Ab dem Punkt, wo es starke KI nicht nur in der Theorie gibt,
sondern auch in der Praxis, funktioniert das spätestens nicht mehr,
da die KI dann selbst erfinderische Tätigkeiten vornehmen kann, welche vom 
Menschen nicht zu unterscheiden sind.
\subsection{Gewerbliche Anwendbarkeit}
Eine Erfindung gilt als gewerblich anwendbar, 
wenn ihr Gegenstand auf irgendeinem gewerblichen
Gebiet einschließlich der Landwirtschaft hergestellt 
oder benutzt werden kann (§ 5 PatG).
Das Erfindungen, 
welche mit/durch KI entstanden sind gewerblich anwendbar sind ist durchaus möglich.
KI-Systeme wie AlphaFold von DeepMind haben Fortschritte 
in der Medikamentenforschung ermöglicht, 
indem sie die 3D-Struktur von Proteinen präzise vorhersagen \footcite{AlphaFold2024}.
\\




