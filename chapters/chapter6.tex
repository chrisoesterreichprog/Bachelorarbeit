\chapter{Fazit und Ausblick\label{cha:chapter7}}

\section{Zusammenfassung\label{sec:summary}}

Die Erfindungen in denen KI mitwirkt oder große
Anteile am Entstehungsprozess übernimmt häufen sich.
Das deutsche und die internationalen Patentämter 
bekommen zunehmend KI-Patente als Anmeldungen eingereicht.
Es ist ausführlich geklärt worden, dass vor dem deutschen 
Patentamt auch KI-Erfindungen angemeldet werden können.
Die KI-Erfindungen dürfen jedoch nicht auf die KI
sondern auf den Nutzer laufen, welcher dann die Rechte und
Pflichten des Erfinders übernimmt. Auch Computerprogramme
können patentiert werden, wenn sie technisch sind und
ein technisches Problem lösen. Sodass einem KI-Patent
in Form eines Computerprogrammes nicht viel im Wege steht.
In dieser Arbeit wurde jedoch auch beleuchtet, dass
eine KI mit unspezifischen Prompts schnell eine für einen 
Fachmann naheliegende Erfindung generieren kann, die
dann nicht mehr patentierbar ist. Außerdem ist die Neuheit
für ein KI-Modell nicht beurteilbar, da es immer einen historischen
Stand aufweist und den aktuellen Stand der Technik nicht einsehen 
kann. Aufgrund dessen reichen die derzeitigen Gesetzgebungen in Deutschland
aus, die Patentierbarkeit von KI-Erfindungen zu bewerten. Es ist nicht 
trivial, eine patentierbare Erfindung mit einer KI zu erschaffen,
ohne das Wissen eines Fachmannes zu haben. Durch das Wissen eines
Fachmannes kann jedoch die erfinderische Tätigkeit und Neuheit
einer KI-Erfindung geprüft und gegebenenfalls angepasst werden.
So bleibt eine gewisse erfinderische Tätigkeit derzeit immer beim '
Menschen und die KI kann nur unterstützend wirken.

\section{Problems Encountered\label{sec:problems}}

Die größte Herausforderung bei dieser Arbeit stellt
die Aktualität dieses Themas dar und die 
schnelle Weiterentwicklung von KI-Systemen.
Die Rechtssprechung kann nicht immer mit der 
technologischen Entwicklung im Bereich KI
Schritt halten und es ist schwer
klare Rechtssprechungen zu finden. Der Fall
"DABUS" ist noch nicht vollständig durch das 
deutsche Patentverfahren
gegangen, obwohl die Erfinderfrage jetzt vom
BGH geklärt wurde. Aufgrund der Aktualität des Themas gibt
es generell wenig Rechtssprechungen, die sich auf
KI-Systeme beziehen. Im Gegensatz zu den Rechtssprechungen
zu Computerprogrammen, die schon seit den 80er Jahren
bestehen, aber auch einen gewissen Spielraum lassen,
in der Patentfähigkeit eines Programmes durch die
"als solche" Formulierung. KI-Systeme
entwickeln sich ständig weiter und während dieser Arbeit 
ist ChatGPT von der Version 3 auf die Version 4o und mittlerweile
o1 gewechselt. Es könnte sein, dass der hypothetische
Patentantrag, wenn er von einer Version o1 entwickelt 
worden wäre, nicht mehr als für
einen Fachmann naheliegend betrachtet wäre und diese Version
auch eine Erfindung erstellt hätte, die die Anforderungen an
die Neuheit erfüllt. Diese Arbeit
kann einen guten Überblick über die aktuelle Rechtslage 
und für schwache KIs eine Übersicht der Patentierbarkeit geben.
Jedoch werden Anpassungen der Gesetzeslage schnell 
folgen müssen um mit der Geschwindigkeit des Fortschritts von
KIs mitzuhalten und jetzt noch gängige Praxis ist in 
einigen Jahren vielleicht schon überholt.
Auch die Erstellung eines hypothetischen Patentantrages
ist nicht trivial, da die vollständige Offenbarung
Fragen aufwirft im Hinblick auf KI-generierte 
Erfindungen. Außerdem kann die Erfidnung je nach 
Input komplett unterschiedlich ausfallen, obwohl 
der Input für einen Menschen kaum unterscheidbar ist.
So führen der Input "Fasse mir die Erfindung zusammen"
und "Erstelle mir eine Zusammenfassung der Erfindung" zu
unterschiedlichen Outputs. Da eine KI-Modell wie 
ChatGPT auch auf den Inputs von Usern lernt,
kann es sein, dass der Output sich abhängig vom 
Zeitpunkt der Eingabe deutlich unterscheidet, was die
Ergebnisse des hypothetischen Patentantrages in Frage 
stellt. Es ist davon auszugehen, dass auch mit denselben 
Prompts patentierbare Erfindungen entstehen können, 
die eine erfinderische Höhe aufweisen, wenn sie zu 
einem anderen Zeitpunkt gestellt werden. 

\section{Ausblick\label{sec:outlook}}

Die derzeitige rechtliche Lage wird durch die 
stetige Weiterentwicklung von KI-Systemen in Frage
gestellt. Das fortwährende Verschwimmen der 
Grenze zwischen menschlicher und maschineller Kreativität
führt dazu, dass die Anforderungen an die Bewertung von 
erfinderischer Tätigkeit und Neuheit neu definiert
werden müssen. Es wird aufgrund noch größerer Datenmengen die einer 
KI zur Verfügung stehen und noch kmplexeren Algorithmen
immer schwieriger den kreativen Prozess einer KI 
nachzuvollziehen. Erfindungen werden entstehen, welche
auch für einen Fachmann nicht naheliegend sind, jedoch 
technisch sehr naheliegend durch den Einsatz von KI.
Dann wird es nötig sein, dass auch der Fachmann zu einer
KI greift, um den aktuellen Stand der Technik identifizieren
zu können. 
Zukünftige Entwicklungen in der Patentrechtspraxis werden zeigen, 
wie eigenständig eine KI werden muss, 
um als juristische Person anerkannt zu werden. 
Die Arbeit zeigt auf, dass klare Regelungen notwendig sind, 
um auch in Zukunft Innovationen zu schützen 
und dabei das Gleichgewicht zwischen 
menschlicher Erfindung und KI-Unterstützung zu wahren.
Starke KI wird zwangsläufig irgendwann dazu 
führen, dass weitreichende Änderungen in der deutschen
Gesetzgebung vorgenommen werden müssen. 
Spätestens ab dem Zeitpunkt der starken KI ändert sich 
das gesamte Patentsystem und es werden weitere 
Maßnahmen wie die Einführung eines Roboterrechts oder
anderer spezifischer Gesetzgebungen für KI folgen. 
Auch eine Einführung einer sog. "elektronischen
Person" könnte in Betracht gezogen werden, um
Rechte für KI Systeme zu schaffen.
Starke KI braucht einen gesonderten Platz im 
Patentrecht, um als gesondert von menschlichen Erfindern
im Innovationsprozess aufzutreten.
Es muss eine Balance zwischen dem Schutz menschlicher Kreativität 
und der Förderung von technologischer Innovation geschaffen werden
um sowohl die Vorteile von KI zu wahren, als 
auch Fairness gegenüber menschlichen Erindern zu 
gewährleisten. Starke KI die am Fließband neue
Erfindungen generiert, wird die Welt wie wir sie 
kennen grundlegend verändern und für komplexe Probleme
unserer Zeit Lösungen finden die wir nicht für möglich
halten aus heutiger Sicht. Es könnte gut dazu kommen,
dass eine starke KI den Menschen vollkommen
ersetzt und die Welt in eine neue Ära führt, in der
KI die treibende Kraft hinter Innovationen ist.
Menschliche Erfindungen weisen dann nicht mehr die 
nötige erfinderische Höhe auf um mit einer starken
KI mithalten zu können. Es ist mithin sinnvoll sich die 
Vorschläge einer starken KI anzuhören, wie sie sich
selber im Patentrecht einordnen würde und ob menschliche
Erfindungen noch eine perspektive haben. Jedoch wird
es nicht von einem auf den anderen Moment eine starke KI
geben, sondern die Entwicklung wird schrittweise erfolgen
und es wird genug Zeit geben, um die Gesetzgebung so
lange immer weiter anzupassen,
bis der Punkt der menschlichen Erindungen Überschritten wurde.


Diese Punkte machen deutlich, 
dass der rechtliche Umgang mit KI und deren Rolle 
im Innovationsprozess einer fortlaufenden Anpassung bedarf, 
um sowohl technologische 
als auch juristische und gesellschaftlichen
Anforderungen gerecht zu werden. Fortwährend werden
neue Gesetzgebungen den Weg leiten und alte Leitsätze 
in Frage gestellt werden um mit dem Fortschritt mitzuhalten.




