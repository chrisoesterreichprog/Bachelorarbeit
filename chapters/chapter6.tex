\chapter{Fazit und Ausblick\label{cha:chapter7}}

\section{Zusammenfassung\label{sec:summary}}

Die Arbeit zeigt, 
dass die fortlaufende Entwicklung von KI neue Herausforderungen 
für das bestehende Patentrecht mit sich bringt, 
insbesondere im Hinblick auf die Patentierbarkeit 
von durch KI generierten Computerprogrammen. 
KI kann mittlerweile Aufgaben übernehmen, die 
traditionell den Einsatz menschlicher Kreativität
und Intelligenz erfordert haben. Das stellt das 
Patentrecht vor Schwierigkeiten bei 
der Fragestellung des Erfinders in Fällen,
wo KI zur Erfindungsgenerierung eingesetzt
wird oder den größten Anteil übernimmt.
Nach der aktuellen Rechtslage 
in Deutschland und Europa dürfen 
nur natürliche Personen als Erfinder benannt werden. 
Der Nutzer der KI gilt als der Erfinder, 
da die KI bislang nur als Werkzeug betrachtet wird, 
das durch einen Menschen bedient wird.
Diskussionen, 
inwieweit KI tatsächlich als Erfinder auftreten könnte,
nehmen zu. 
Das Beispiel von DABUS zeigt das KI durchaus viele
Anforderungen an die Patentierbarkeit erfüllen
kann.

Die Arbeit hat ebenfalls deutlich gemacht, 
dass das deutsche Patentrecht bereits Möglichkeiten hat, 
Patente für Computerprogramme zu erteilen, 
insofern diese eine technische Problemlösung darstellen. 
Die sog. "als solche" Regelung, welche Computerprogramme
von der Patentierbarkeit ausschließt,
gilt nicht wenn ein Computerprogramm eine technische Wirkung 
aufweist.
Damit besteht die Möglichkeit, 
Computerprogramme, die durch KI erstellt wurden, 
zu patentieren, solange sie eine technische Wirkung aufweisen 
und die formellen Voraussetzungen,
wie der Angabe des Nutzer als Erfinder, erfüllen.
Dies ist besonders spannend, da KI Computerprogramme
in den technischen Bereichen, wie 
der Automatisierung, der Bildverarbeitung oder dem Energiemanagement
erzeugen kann.
\section{Problems Encountered\label{sec:problems}}

Die größte Herausforderung bei dieser Arbeit stellt
die Aktualität dieses Themas dar und die 
schnelle Weiterentwicklung von KI-Systemen.
Die Rechtssprechung kann nicht immer mit der 
technologischen Entwicklung im Bereich KI
Schritt halten und es ist schwer
klare Rechtssprechungen zu finden. Der Fall
"DABUS" ist noch nicht vollständig durch das 
deutsche Patentverfahren
gegangen, obwohl die Erfinderfrage jetzt vom
BGH geklärt wurde. Aufgrund der Aktualität des Themas gibt
es generell wenig Rechtssprechungen, die sich auf
KI-Systeme beziehen. Im Gegensatz zu den Rechtssprechungen
zu Computerprogrammen, die schon seit den 80er Jahren
bestehen, aber auch einen gewissen Spielraum lassen,
in der Definition der Technizität eines Programmes. KI-Systeme
entwickeln sich ständig weiter und während dieser Arbeit 
ist ChatGPT von der Version 3 auf die Version 4o und mittlerweile
o1 gewechselt. Es könnte sein, dass der hypothetische
Patentantrag, wenn er von einer Version o1 entwickelt 
worden wäre, nicht mehr als für
einen Fachmann naheliegend betrachtet wird. Diese Arbeit
kann einen guten Überblick über die aktuelle Rechtslage 
und für schwache KIs eine Übersicht der Patentierbarkeit geben.
Jedoch werden Anpassungen der Gesetzeslage schnell 
folgen müssen um mit der Geschwindigkeit des Fortschritts von
KIs mitzuhalten und jetzt noch gängige Praxis ist in 
einigen Jahren vielleicht schon überholt.
Auch die Erstellung eines hypothetischen Patentantrages
ist nicht trivial, da die vollständige Offenbarung
Fragen aufwirft im Hinblick auf KI-generierte 
Erfindungen. Außerdem kann die Erfidnung je nach 
Input komplett unterschiedlich ausfallen, obwohl 
der Input für einen Menschen kaum unterscheidbar ist.
So führen der Input "Fasse mir die Erfindung zusammen"
und "Erstelle mir eine Zusammenfassung der Erfindung" zu
unterschiedlichen Outputs. Da eine KI-Modell wie 
ChatGPT auch auf den Inputs von Usern lernt,
kann es sein, dass der Output sich abhängig vom 
Zeitpunkt der Eingabe deutlich unterscheidet, was die
Ergebnisse des hypothetischen Patentantrages in Frage 
stellt. Es ist davon auszugehen, dass auch mit denselben 
Prompts patentierbare Erfindungen entstehen können, 
die eine erfinderische Höhe aufweisen. 

\section{Ausblick\label{sec:outlook}}

Die derzeitige rechtliche Lage wird durch die 
stetige Weiterentwicklung von KI-Systemen in Frage
gestellt. Das fortwährende Verschwimmen der 
Grenze zwischen menschlicher und maschineller Kreativität
dazu, dass die Anforderungen an die Bewertung von 
erfinderischer Tätigkeit und Neuheit neu definiert
werden müssen. 
Zukünftige Entwicklungen in der Patentrechtspraxis werden zeigen, 
wie weit die Automatisierung und Eigenständigkeit von KIs voranschreiten muss, 
um als juristische Person anerkannt zu werden. 
Die Arbeit zeigt auf, dass klare Regelungen notwendig sind, 
um auch in Zukunft Innovationen zu schützen 
und dabei das Gleichgewicht zwischen 
menschlicher Erfindung und KI-Unterstützung zu wahren.
Es wird aufgrund noch größerer Datenmengen die einer 
KI zur Verfügung stehen und noch kmplexeren Algorithmen
noch schwieriger den kreativen Prozess einer KI 
nachzuvollziehen. Erfindungen werden entstehen, welche
auch für einen Fachmann nicht naheliegend sind, jedoch 
technisch sehr naheliegend durch den Einsatz von KI.
Dann wird es nötig sein, dass auch der Fachmann zu einer
KI greift, um den aktuellen Stand der Technik identifizieren
zu können. 

Starke KI wird zwangsläufig irgendwann dazu 
führen, dass weitreichende Änderungen in der 
Gesetzgebung vorgenommen werden müssen. Dabei steht 
vorallem die Frage im Vordergrund, ab wann eine 
KI als eigenständige Rechtsperson angesehen werden
kann. Spätestens ab diesem Zeitpunkt ändert sich 
das gesamte Patentsystem und es werden weitere 
Maßnahmen wie die Einführung eines Roboterrechts oder
andere spezifische Gesetzgebungen für KI folgen. 
Auch eine Einführung einer sog. "elektronischen
Person" könnte in Betracht gezogen werden, um
Rechte für KI Systeme zu schaffen.
Starke KI braucht einen gesonderten Platz um 
Patentrecht, um als unabhängige Akteure 
im Innovationsprozess aufzutreten.
Es muss eine Balance zwischen dem Schutz menschlicher Kreativität 
und der Förderung von technologischer Innovation geschaffen werden
um sowohl die Vorteile von KI voll auszuschöpfen, als 
auch Fairness den gegenüber menschlichen Erindern zu 
gewährleisten.

Diese Punkte machen deutlich, 
dass der rechtliche Umgang mit KI und deren Rolle 
im Innovationsprozess einer fortlaufenden Anpassung bedarf, 
um sowohl technologische 
als auch juristische und gesellschaftlichen
Anforderungen gerecht zu werden.
