\thispagestyle{empty}
\vspace*{1.0cm}

\begin{center}
    \textbf{Abstract}
\end{center}

\vspace*{0.5cm}

\noindent

The rapid development of Artifical Intelligence (AI)
poses new challenges for patent law. 
In particular, the patentability of computer programs generated by AI 
is an exciting topic, 
as traditionally only human inventors are considered in patent law. 
This paper analyzes the case law and challenges 
the patenting of AI-generated computer programs in German patent law. 
in German patent law. 
The novelty, inventive step and industrial applicability as well as the question of 
as well as the question of the designation of the inventor in the context of AI. 
In addition, the problem of the lack of legal recognition of AI as an inventor and possible 
possible approaches for future changes to the law. 
A hypothetical patent application for an invention developed by AI in the field of 
in the field of IOT serves as a case study, 
to illustrate the practical application of the current law. 
Finally, the paper provides an outlook 
on possible developments in patent law 
with regard to the handling of AI-generated inventions.
